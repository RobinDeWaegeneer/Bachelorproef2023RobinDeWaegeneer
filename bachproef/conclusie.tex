%%=============================================================================
%% Conclusie
%%=============================================================================

\chapter{Conclusie}%
\label{ch:conclusie}

% TODO: Trek een duidelijke conclusie, in de vorm van een antwoord op de
% onderzoeksvra(a)g(en). Wat was jouw bijdrage aan het onderzoeksdomein en
% hoe biedt dit meerwaarde aan het vakgebied/doelgroep? 
% Reflecteer kritisch over het resultaat. In Engelse teksten wordt deze sectie
% ``Discussion'' genoemd. Had je deze uitkomst verwacht? Zijn er zaken die nog
% niet duidelijk zijn?
% Heeft het onderzoek geleid tot nieuwe vragen die uitnodigen tot verder 
%onderzoek?

%%\lipsum[76-80]

De literatuurstudie heeft een duidelijk beeld geschept van het brede landschap aan diverse tools en applicaties die reeds beschikbaar zijn. Deze blijken niet meteen aangepast voor personen met een mentale beperking of nood aan een prikkelarme weergave zoals het weglaten van meldingen en advertenties die voor afleiding kunnen zorgen. Locatiebepaling zorgt ervoor dat punt A niet noodzakelijk moeten worden geïdentificeerd door de gebruiker, maar dit is enkel geldig als deze vertrekt van zijn huidige locatie en niet voor het vooraf plannen van een route vanaf een willekeurig punt A. Sommige applicaties bieden zeker opties via speech-to-text om het zelf intypen van een locatie A of B te omzeilen. Het opzoeken van bepaalde points of interest is nuttig wanneer een exact adres niet gekend is, maar de gebruiker zich wil richten op een bepaalde activiteit (vrije tijd, transportmiddel, ...).  

De requirementsanalyse leverde een short list op van twee tools die van naderbij zijn geanalyseerd: Mapbox en Google Maps. Hoewel beide applicaties/tools toegevoegde waarde hebben en bovendien heel hard op mekaar lijken, kan er toch geconcludeerd worden dat de applicatie Mapbox meer geschikt is voor ontwikkelaars. Ze is meer aangepast voor het ontwikkelen van eigen noden in een applicatie op maat van de doelgroep. Google Maps daarentegen is een meer kant-en-klare oplossing indien er geen technische kennis is voor het ontwikkelen van een eigen applicatie. Beide kunnen ze waarde bieden in het navigeren van punt A naar punt B.

Augmented Reality (AR) levert een goede extra laag bovenop de standaard navigatie die kan helpen bij het navigeren van de mensen van de doelgroep. De duidelijke signalen die AR teruggeeft kunnen een cruciale stap zijn in tijdswinst en duidelijkheid voor de gebruiker.

Voor toekomstig onderzoek wordt aanbevolen een versie te ontwikkelen die een aantal standaardtools van de doelgroep integreert, zoals Sclera pictogrammen wanneer bepaalde informatie wordt gevraagd of weergegeven. Voor mensen met autismespectrum moet de aangepaste applicatie toelaten te kiezen voor het weglaten van bepaalde geluiden of meldingen die de gebruiker als storend ervaart. 