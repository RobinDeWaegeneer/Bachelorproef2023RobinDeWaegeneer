%%=============================================================================
%% Methodologie
%%=============================================================================

\chapter{\IfLanguageName{dutch}{Methodologie}{Methodology}}%
\label{ch:methodologie}

%% TODO: In dit hoofstuk geef je een korte toelichting over hoe je te werk bent
%% gegaan. Verdeel je onderzoek in grote fasen, en licht in elke fase toe wat
%% de doelstelling was, welke deliverables daar uit gekomen zijn, en welke
%% onderzoeksmethoden je daarbij toegepast hebt. Verantwoord waarom je
%% op deze manier te werk gegaan bent.
%% 
%% Voorbeelden van zulke fasen zijn: literatuurstudie, opstellen van een
%% requirements-analyse, opstellen long-list (bij vergelijkende studie),
%% selectie van geschikte tools (bij vergelijkende studie, "short-list"),
%% opzetten testopstelling/PoC, uitvoeren testen en verzamelen
%% van resultaten, analyse van resultaten, ...
%%
%% !!!!! LET OP !!!!!
%%
%% Het is uitdrukkelijk NIET de bedoeling dat je het grootste deel van de corpus
%% van je bachelorproef in dit hoofstuk verwerkt! Dit hoofdstuk is eerder een
%% kort overzicht van je plan van aanpak.
%%
%% Maak voor elke fase (behalve het literatuuronderzoek) een NIEUW HOOFDSTUK aan
%% en geef het een gepaste titel.

%%\lipsum[21-25]
De methodologie van dit onderzoek naar navigatiemethoden voor mensen met een mentale beperking is opgedeeld in drie belangrijke fasen, elk essentieel voor het bereiken van de onderzoeksdoelstellingen: een literatuurstudie, een requirementsanalyse en de uitwerking van een Proof of Concept (PoC) inclusief een praktijkexperiment.\newline

De eerste fase bestaat uit een uitgebreide literatuurstudie waarin de basisbegrippen van navigatie en ondersteunende technologieën in kaart worden gebracht, evenals de concrete definitie van de impact van een mentale beperking op het gebruik hiervan.  Door bestaande navigatiemethoden en -tools te bestuderen, evalueren we hun toepasbaarheid en effectiviteit voor onze specifieke doelgroep. De hoofdvraag omvat diverse deelaspecten: kan locatiebepaling helpen om zelfstandige input van de gebruikers te omzeilen, kan de integratie van AR een toegevoegde waarde bieden omtrent veiligheidsgevoel en comfort en wat met de specifieke communicatiemogelijkheden voor mensen met een beperking. Deze fase is cruciaal voor het beantwoorden van de deelvraag over het huidige landschap van navigatiehulpmiddelen, meerbepaald het identificeren van eventuele lacunes in de bestaande technologieën. Het eindresultaat is een mapping die kan afgetoetst worden in functie van de gebruikersnoden en als input dient voor de requirementsanalyse.

\subsection*{Requirementsanalyse}

Na het opbouwen van een theoretisch kader, start de tweede fase met een vergelijkende analyse van de geïdentificeerde navigatiemethoden. Hierbij maken we gebruik van de MoSCoW-methode om prioriteit te geven aan verschillende functies en mogelijkheden die essentieel zijn voor onze doelgroep. Ze houdt rekening met een aantal functies, integraties en mogelijke werkwijzen zoals hieronder opgelijst:

\begin{enumerate}
    \item \textbf{Functies}: Een lijst van alle nodige functies in de app (location tracking, text-to-speech, voorleesfunctie, language models, visualisatietypes, Sclera pictogrammen)
    \item \textbf{Integraties}: Een omschrijving van mogelijke AR integraties (Google Maps, Mapbox, Koombea, AR City, Sygic)
    \item \textbf{Werkwijze en denkpatronen}: Een selectie van enkele mogelijke manieren waarmee interactie kan geïnitieerd worden inclusief AR (kleuren, geluiden, patronen, figuren, ...)
\end{enumerate}
    
    We bepalen welke criteria 'Must Have', 'Should Have', 'Could Have' en 'Won't Have' zijn. Alleen de methoden die voldoen aan de 'Must Have'-criteria zullen worden geselecteerd voor verdere evaluatie. Dit zorgt ervoor dat de gekozen methoden niet alleen technisch haalbaar zijn, maar ook maximaal aansluiten bij de behoeften van mensen met een mentale beperking. Zij vormen de basis van de proof of concept.
    
    \subsection*{Proof of Concept en praktijktest}
    
    In de derde en laatste fase ontwikkelen we een proof of concept door een testapplicatie te bouwen met behulp van de geselecteerde navigatiemethoden. De POC zal worden uitgewerkt op een smartphone voor een aantal specifieke routes is. Deze applicatie, ontwikkeld in React Native, stelt ons in staat om de functionaliteit en gebruikerservaring van elke methode in de praktijk te testen met de doelgroep. We verzamelen vooreerst data over de bruikbaarheid, toegankelijkheid en effectiviteit door middel van directe observaties en gebruikersfeedback. Bovendien voeren we een data-analyse uit van de tijd nodig voor de verplaatsing van punt A naar punt B, zonder (nulmeting) en met gebruik van de applicatie. Hierdoor verschaffen we zowel inzicht in de effectiviteit van de mobiliteitstool, namelijk het verminderen van de benodigde tijd voor een verplaatsing, als in het gebruiksgemak.
    
    \subsection*{Evaluatie en aanbevelingen}
    
    Evaluatie van de moeilijkheidsgraad zoals ervaren door de gebruikers zal samen met de uitkomsten van deze test toelaten de meest geschikte navigatiemethode(s) te identificeren die best voldoen aan de behoeften van onze doelgroep, gebaseerd op de criteria vastgesteld in fase 2. Deze conclusies zullen essentieel zijn voor het aanbevelen van specifieke navigatietechnologieën en -benaderingen die kunnen worden geïmplementeerd in mobiliteitstools voor mensen met een mentale beperking, waarbij we streven naar het verbeteren van hun zelfstandigheid en kwaliteit van leven.

