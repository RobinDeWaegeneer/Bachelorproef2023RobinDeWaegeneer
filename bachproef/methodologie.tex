%%=============================================================================
%% Methodologie
%%=============================================================================

\chapter{\IfLanguageName{dutch}{Methodologie}{Methodology}}%
\label{ch:methodologie}

%% TODO: In dit hoofstuk geef je een korte toelichting over hoe je te werk bent
%% gegaan. Verdeel je onderzoek in grote fasen, en licht in elke fase toe wat
%% de doelstelling was, welke deliverables daar uit gekomen zijn, en welke
%% onderzoeksmethoden je daarbij toegepast hebt. Verantwoord waarom je
%% op deze manier te werk gegaan bent.
%% 
%% Voorbeelden van zulke fasen zijn: literatuurstudie, opstellen van een
%% requirements-analyse, opstellen long-list (bij vergelijkende studie),
%% selectie van geschikte tools (bij vergelijkende studie, "short-list"),
%% opzetten testopstelling/PoC, uitvoeren testen en verzamelen
%% van resultaten, analyse van resultaten, ...
%%
%% !!!!! LET OP !!!!!
%%
%% Het is uitdrukkelijk NIET de bedoeling dat je het grootste deel van de corpus
%% van je bachelorproef in dit hoofstuk verwerkt! Dit hoofdstuk is eerder een
%% kort overzicht van je plan van aanpak.
%%
%% Maak voor elke fase (behalve het literatuuronderzoek) een NIEUW HOOFDSTUK aan
%% en geef het een gepaste titel.

%%\lipsum[21-25]
De methodologie van dit onderzoek naar navigatiemethoden voor mensen met een mentale beperking is opgedeeld in drie belangrijke fasen, elk essentieel voor het bereiken van de onderzoeksdoelstellingen: een literatuurstudie, een requirementsanalyse en de uitwerking van een Proof of Concept (PoC). Dankzij de requirementsanalyse is een long list van methoden ingekort tot een short list van diegenen best in aanmerking konden komen voor de doelgroep. De vergelijkende studie van deze shortlist is besproken in de resultatenverwerking en heeft aanleiding gegeven tot enkele gerichte conclusies, waarmee de scriptie is afgerond onder de vorm van enkele aanbevelingen.

\subsection*{Mapping}
De eerste fase heeft bestaan uit een uitgebreide literatuurstudie waarin de basisbegrippen van navigatie en ondersteunende technologieën in kaart worden gebracht, evenals de concrete definitie van de impact van een mentale beperking op het gebruik hiervan.  Door bestaande navigatiemethoden en -tools te bestuderen, is hun toepasbaarheid en effectiviteit voor onze specifieke doelgroep geëvalueerd. De hoofdvraag is daarom opgesplitst in diverse deelaspecten. Kan locatiebepaling helpen om zelfstandige input van de gebruikers te omzeilen ? Kan de integratie van AR een toegevoegde waarde bieden omtrent veiligheidsgevoel en comfort ? Welke specifieke communicatiemogelijkheden voor mensen met een beperking zijn bruikbaar? Deze fase is cruciaal geweest voor het beantwoorden van de deelvraag over het huidige landschap van navigatiehulpmiddelen, meerbepaald het identificeren van eventuele lacunes in de bestaande technologieën. Het eindresultaat is een mapping die kan afgetoetst worden in functie van de gebruikersnoden en als input heeft gediend voor de requirementsanalyse.

\subsection*{Requirementsanalyse en Long List}

Na het opbouwen van een theoretisch kader, is gestart met de tweede fase, namelijk een vergelijkende studie van de geïdentificeerde navigatiemethoden. Hierbij is gebruik gemaakt van de MoSCoW-methode om prioriteit te geven aan verschillende requirements die essentieel zijn voor de doelgroep. Ze heeft rekening gehouden met een aantal functies, integraties en mogelijke werkwijzen zoals hieronder opgelijst:

\begin{enumerate}
    \item \textbf{Functies}: Een lijst van alle nodige functies in de app bv. location tracking, text-to-speech, voorleesfunctie, language models, visualisatietypes, Sclera pictogrammen
    \item \textbf{Integraties}: Een omschrijving van mogelijke AR integraties
    \item \textbf{Werkwijze en denkpatronen}: Een selectie van enkele mogelijke manieren waarmee interactie kan geïnitieerd worden inclusief AR (kleuren, geluiden, patronen, figuren, $\ldots$)
\end{enumerate}

Er is bepaald welke criteria 'Must Have', 'Should Have', 'Could Have' en 'Won't Have' zijn. Alleen de mogelijkheden die voldoen aan de 'Must Have'-criteria werden geselecteerd voor verdere evaluatie. Dit heeft ervoor gezorgd dat de gekozen methoden niet alleen technisch haalbaar zijn, maar ook maximaal aansluiten bij de behoeften van mensen met een mentale beperking. Zij hebben de basis gevormd van de proof of concept.

\subsection*{Proof of Concept en Short List}

In de derde en laatste fase is een proof of concept ontwikkeld door twee geselecteerde navigatiemethoden te evalueren voor de opbouw van een testapplicatie. De POC is uitgewerkt voor een smartphone voor een aantal specifieke routes. Deze applicatie is ontwikkeld in React Native om de functionaliteit en gebruikerservaring te evalueren in functie van de doelgroep. Het gedeelte waarin data wordt verzameld over bruikbaarheid, toegankelijkheid en effectiviteit door middel van directe observaties en gebruikersfeedback, en waar een data-analyse wordt uitgevoerd van de tijd nodig voor verplaatsing van punt A naar punt B zonder en met gebruik van de applicatie, is vervangen door een gedetailleerde evaluatie van de navigatiemethoden. Deze evaluatie omvat de identificatie van de meest geschikte navigatiemethoden op basis van de criteria vastgesteld in de tweede fase, en de aanbevelingen van specifieke technologieën en benaderingen die het beste voldoen aan de behoeften van de doelgroep. Dit houdt in dat de nadruk verschuift van het simpelweg meten van reistijd en gebruikerservaring naar een bredere analyse en aanbeveling van navigatiemethoden en technologieën die het meest effectief zijn voor mensen met een mentale beperking.

\subsection*{Resultatenverwerking, evaluatie en aanbevelingen}

De uitkomst van deze vergelijking is gebruikt om de meest geschikte navigatiemethode(s) te identificeren die best voldoen aan de behoeften van onze doelgroep, gebaseerd op de criteria vastgesteld in fase 2. Daaruit zijn essentiële conclusies gevolgd nodig voor het aanbevelen van specifieke navigatietechnologieën en -benaderingen die kunnen worden geïmplementeerd in mobiliteitstools voor mensen met een mentale beperking, waarbij gestreefd wordt naar het verbeteren van hun zelfstandigheid en kwaliteit van leven.