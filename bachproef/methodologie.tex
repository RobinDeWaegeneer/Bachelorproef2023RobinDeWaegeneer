%%=============================================================================
%% Methodologie
%%=============================================================================

\chapter{\IfLanguageName{dutch}{Methodologie}{Methodology}}%
\label{ch:methodologie}

%% TODO: In dit hoofstuk geef je een korte toelichting over hoe je te werk bent
%% gegaan. Verdeel je onderzoek in grote fasen, en licht in elke fase toe wat
%% de doelstelling was, welke deliverables daar uit gekomen zijn, en welke
%% onderzoeksmethoden je daarbij toegepast hebt. Verantwoord waarom je
%% op deze manier te werk gegaan bent.
%% 
%% Voorbeelden van zulke fasen zijn: literatuurstudie, opstellen van een
%% requirements-analyse, opstellen long-list (bij vergelijkende studie),
%% selectie van geschikte tools (bij vergelijkende studie, "short-list"),
%% opzetten testopstelling/PoC, uitvoeren testen en verzamelen
%% van resultaten, analyse van resultaten, ...
%%
%% !!!!! LET OP !!!!!
%%
%% Het is uitdrukkelijk NIET de bedoeling dat je het grootste deel van de corpus
%% van je bachelorproef in dit hoofstuk verwerkt! Dit hoofdstuk is eerder een
%% kort overzicht van je plan van aanpak.
%%
%% Maak voor elke fase (behalve het literatuuronderzoek) een NIEUW HOOFDSTUK aan
%% en geef het een gepaste titel.

%%\lipsum[21-25]
Dit onderzoek over navigatiemethoden gericht voor mensen met een mentale beperking is opgedeeld in drie belangrijke fasen, elk essentieel voor het bereiken van de onderzoeksdoelstellingen. De eerste fase bestaat uit een uitgebreide literatuurstudie waarin we de basisbegrippen van navigatie en ondersteunende technologieën voor mensen met een mentale beperking onderzoeken. We bestuderen bestaande navigatiemethoden en -tools, en evalueren hun toepasbaarheid en effectiviteit voor onze specifieke doelgroep. Deze fase is cruciaal voor het beantwoorden van de deelvraag over het huidige landschap van navigatiehulpmiddelen en het identificeren van eventuele lacunes in de bestaande technologieën. Na het opbouwen van een theoretisch kader, start de tweede fase met een vergelijkende analyse van de geïdentificeerde navigatiemethoden. Hierbij maken we gebruik van de MoSCoW-methode om prioriteit te geven aan verschillende functies en mogelijkheden die essentieel zijn voor onze doelgroep. We bepalen welke criteria 'Must Have', 'Should Have', 'Could Have' en 'Won't Have' zijn. Alleen de methoden die voldoen aan de 'Must Have'-criteria zullen worden geselecteerd voor verdere evaluatie. Dit zorgt ervoor dat de gekozen methoden niet alleen technisch haalbaar zijn, maar ook maximaal aansluiten bij de behoeften van mensen met een mentale beperking.In de derde en laatste fase ontwikkelen we een proof of concept door een testapplicatie te bouwen met behulp van de geselecteerde navigatiemethoden. Deze applicatie, ontwikkeld in React Native, stelt ons in staat om de functionaliteit en gebruikerservaring van elke methode in de praktijk te testen met de doelgroep. We verzamelen data over de bruikbaarheid, toegankelijkheid en effectiviteit door middel van directe observaties en gebruikersfeedback. Bovendien voeren we een data-analyse uit van de verzamelde informatie, waarbij we specifiek kijken naar de tijd die nodig is voor navigatie en de moeilijkheidsgraad zoals ervaren door de gebruikers. De uitkomsten van deze tests zullen ons helpen om de meest geschikte navigatiemethode te identificeren die geïmplementeerd kan worden in tools bedoeld voor mensen met een mentale beperking. De resultaten van deze tests en analyses zullen leiden tot de ontwikkeling van een conclusie die de effectiviteit van elke geteste navigatiemethode beoordeelt. We zullen bepalen welke methode(s) het best voldoen aan de behoeften van onze doelgroep, gebaseerd op de criteria vastgesteld in fase 2. Deze conclusies zullen essentieel zijn voor het aanbevelen van specifieke navigatietechnologieën en -benaderingen die kunnen worden geïmplementeerd in tools voor mensen met een mentale beperking, waarbij we streven naar het verbeteren van hun zelfstandigheid en kwaliteit van leven.

