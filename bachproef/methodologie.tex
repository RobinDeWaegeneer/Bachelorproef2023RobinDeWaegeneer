%%=============================================================================
%% Methodologie
%%=============================================================================

\chapter{\IfLanguageName{dutch}{Methodologie}{Methodology}}%
\label{ch:methodologie}

%% TODO: In dit hoofstuk geef je een korte toelichting over hoe je te werk bent
%% gegaan. Verdeel je onderzoek in grote fasen, en licht in elke fase toe wat
%% de doelstelling was, welke deliverables daar uit gekomen zijn, en welke
%% onderzoeksmethoden je daarbij toegepast hebt. Verantwoord waarom je
%% op deze manier te werk gegaan bent.
%% 
%% Voorbeelden van zulke fasen zijn: literatuurstudie, opstellen van een
%% requirements-analyse, opstellen long-list (bij vergelijkende studie),
%% selectie van geschikte tools (bij vergelijkende studie, "short-list"),
%% opzetten testopstelling/PoC, uitvoeren testen en verzamelen
%% van resultaten, analyse van resultaten, ...
%%
%% !!!!! LET OP !!!!!
%%
%% Het is uitdrukkelijk NIET de bedoeling dat je het grootste deel van de corpus
%% van je bachelorproef in dit hoofstuk verwerkt! Dit hoofdstuk is eerder een
%% kort overzicht van je plan van aanpak.
%%
%% Maak voor elke fase (behalve het literatuuronderzoek) een NIEUW HOOFDSTUK aan
%% en geef het een gepaste titel.

%%\lipsum[21-25]
De methodologie van dit onderzoek naar navigatiemethoden voor mensen met een mentale beperking is opgedeeld in drie belangrijke fasen, elk essentieel voor het bereiken van de onderzoeksdoelstellingen: een literatuurstudie, een requirementsanalyse, long list, short list en de uitwerking van een Proof of Concept (PoC) van de alternatieven op de short list. Vervolgens worden de Proof of Concets geëvalueerd o.b.v. de eerder geformuleerde requirements, en worden er aanbevelingen geformuleerd. Uiteindelijk wordt in de conclusie een synthese gemaakt van de resultaten en worden de onderzoeksvragen beantwoord.

De volgende paragrafen geven een overzicht van de verschillende fasen van het onderzoek en de methoden die zijn gebruikt om de doelstellingen te bereiken.

\subsection*{Literatuurstudie}

De eerste fase heeft bestaan uit een uitgebreide literatuurstudie waarin de basisbegrippen van navigatie en ondersteunende technologieën in kaart werden gebracht, evenals de concrete definitie van de impact van een mentale beperking op het gebruik hiervan. Door bestaande navigatiemethoden en -tools te bestuderen, is hun toepasbaarheid en effectiviteit voor onze specifieke doelgroep geëvalueerd. De hoofdvraag is daarom opgesplitst in diverse deelaspecten, nl. of locatiebepaling kan helpen om zelfstandige input van de gebruikers te omzeilen, of de integratie van AR een toegevoegde waarde kan bieden omtrent veiligheidsgevoel en comfort, en welke specifieke communicatiemogelijkheden voor mensen met een beperking bruikbaar zijn. Deze fase is cruciaal geweest voor het beantwoorden van de deelvraag over het huidige landschap van navigatiehulpmiddelen, meerbepaald het identificeren van eventuele lacunes in de bestaande technologieën. Het eindresultaat is een mapping die kan afgetoetst worden in functie van de gebruikersnoden en als input heeft gediend voor de requirementsanalyse.

\subsection*{Requirementsanalyse, long list en short list}

Na het opbouwen van een theoretisch kader, is gestart met de tweede fase, namelijk een vergelijkende studie van de geïdentificeerde navigatiemethoden. Hierbij is gebruik gemaakt van de MoSCoW-methode om prioriteit te geven aan verschillende requirements die essentieel zijn voor de doelgroep. Er is bepaald welke criteria 'Must Have', 'Should Have', 'Could Have' en 'Won't Have' zijn. Alleen de mogelijkheden die voldoen aan de 'Must Have'-criteria werden geselecteerd voor verdere evaluatie. Dit heeft ervoor gezorgd dat de gekozen methoden niet alleen technisch haalbaar zijn, maar ook maximaal aansluiten bij de behoeften van mensen met een mentale beperking.

De resultaten van deze fase zijn een long list van mogelijke navigatiemethoden en een short list van de meest geschikte methoden die in aanmerking komen voor de ontwikkeling van een Proof of Concept.

\subsection*{Proof of Concept}

% TODO: dit moet je nog aanpassen an wat je werkelijk gedaan hebt!

In de derde en laatste fase is een proof of concept ontwikkeld door twee geselecteerde navigatiemethoden te evalueren voor de opbouw van een testapplicatie. De POC is uitgewerkt voor een smartphone voor een aantal specifieke routes. Deze applicatie is ontwikkeld in React Native om de functionaliteit en gebruikerservaring te evalueren in functie van de doelgroep. Het gedeelte waarin data wordt verzameld over bruikbaarheid, toegankelijkheid en effectiviteit door middel van directe observaties en gebruikersfeedback, en waar een data-analyse wordt uitgevoerd van de tijd nodig voor verplaatsing van punt A naar punt B zonder en met gebruik van de applicatie, is vervangen door een gedetailleerde evaluatie van de navigatiemethoden. Deze evaluatie omvat de identificatie van de meest geschikte navigatiemethoden op basis van de criteria vastgesteld in de tweede fase, en de aanbevelingen van specifieke technologieën en benaderingen die het beste voldoen aan de behoeften van de doelgroep. Dit houdt in dat de nadruk verschuift van het simpelweg meten van reistijd en gebruikerservaring naar een bredere analyse en aanbeveling van navigatiemethoden en technologieën die het meest effectief zijn voor mensen met een mentale beperking.


\subsection*{Resultatenverwerking, evaluatie en aanbevelingen}

De uitkomst van deze vergelijking is gebruikt om de meest geschikte navigatiemethode(s) te identificeren die best voldoen aan de behoeften van onze doelgroep, gebaseerd op de criteria vastgesteld in de requirementsanalyse. Daaruit zijn essentiële conclusies gevolgd nodig voor het aanbevelen van specifieke navigatietechnologieën en -benaderingen die kunnen worden geïmplementeerd in mobiliteitstools voor mensen met een mentale beperking, waarbij gestreefd wordt naar het verbeteren van hun zelfstandigheid en kwaliteit van leven.

\subsection*{Conclusie}

In deze laatste fase worden de resultaten van de PoC geëvalueerd en worden de onderzoeksvragen beantwoord. De conclusie geeft een synthese van de resultaten en de aanbevelingen die voortvloeien uit het onderzoek. Het resultaat is een aanbeveling voor de meest geschikte navigatiemethode voor mensen met een mentale beperking, gebaseerd op de requirementsanalyse en de evaluatie van de PoC.