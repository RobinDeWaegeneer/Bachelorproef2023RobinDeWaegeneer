%%=============================================================================
%% Methodologie
%%=============================================================================

\chapter{\IfLanguageName{dutch}{Methodologie}{Methodology}}%
\label{ch:methodologie}

%% TODO: In dit hoofstuk geef je een korte toelichting over hoe je te werk bent
%% gegaan. Verdeel je onderzoek in grote fasen, en licht in elke fase toe wat
%% de doelstelling was, welke deliverables daar uit gekomen zijn, en welke
%% onderzoeksmethoden je daarbij toegepast hebt. Verantwoord waarom je
%% op deze manier te werk gegaan bent.
%% 
%% Voorbeelden van zulke fasen zijn: literatuurstudie, opstellen van een
%% requirements-analyse, opstellen long-list (bij vergelijkende studie),
%% selectie van geschikte tools (bij vergelijkende studie, "short-list"),
%% opzetten testopstelling/PoC, uitvoeren testen en verzamelen
%% van resultaten, analyse van resultaten, ...
%%
%% !!!!! LET OP !!!!!
%%
%% Het is uitdrukkelijk NIET de bedoeling dat je het grootste deel van de corpus
%% van je bachelorproef in dit hoofstuk verwerkt! Dit hoofdstuk is eerder een
%% kort overzicht van je plan van aanpak.
%%
%% Maak voor elke fase (behalve het literatuuronderzoek) een NIEUW HOOFDSTUK aan
%% en geef het een gepaste titel.

%%\lipsum[21-25]

De methodologie van het onderzoek bestaat uit een aantal belangrijke deelstappen waaronder literatuurstudie, een requirementsanalyse door middel van bevraging van de noden bij de stakeholders, definitie van de software inclusief Augmented Reality (AR) opties, mogelijke hardwarecomponenten voor implementatie, uitwerking van de Proof-of-Concept (POC), een praktijkexperiment in meerdere fasen, het formuleren van de conclusies en de scriptie. De hoofdvraag omvat diverse deelaspecten zoals ``kan locatiebepaling helpen om zelfstandige input van de gebruikers te omzeilen''. Anderzijds zal ook het deelaspect van de specifieke communicatiemogelijkheden voor mensen met een beperking worden geëvalueerd. Tot slot brengt de integratie van AR een toegevoegde waarde omtrent veiligheidsgevoel en comfort, wat moet blijken uit de feedback van de praktijktesten. Na de nodige research online en in bestaande applicaties van huidige aanbieders, is een bevraging van diverse stakeholders gepland. Aan de hand van interviews met Fiola\footnote{\url{https://fiolavzw.be}} (begeleiding mensen met mentale beperking), Tanderuis\footnote{\url{https://www.tanderuis.be}} (gespecialiseerd in autismespectrumstoornis), begeleidingsorganisaties en vakexperten die mensen met een matige mentale beperking begeleiden, worden inzichten verworven over de huidige mogelijkheden en de specifieke noden. Welke zijn de basisvoorwaarden voor een toegankelijke applicatie en welke aangepaste communicatiemiddelen kunnen praktisch ingezet worden. Deze bevraging zal het lastenboek definiëren voor onze POC en mogelijkheden voor samenwerking verduidelijken.
Functionaliteiten van elk deze tools zullen gescreend worden om te bepalen of zij in aanmerking komen voor integratie in de POC.
Voor het evalueren van de mobiliteitstool zullen praktijktesten worden uitgevoerd, waarbij de tijd die nodig is voor de verplaatsing van punt A naar punt B centraal staat. 
Deze testen zullen gebaseerd zijn op tijdmetingen, waarbij zowel een nulmeting zonder gebruik van de applicatie als met de mobiliteitstool zal worden uitgevoerd. We zullen bij de nulmeting ook hun route traceren.
Het doel van deze tests is om inzicht te krijgen in de effectiviteit van de mobiliteitstool in het verminderen van de benodigde tijd voor verplaatsingen en om eventuele verbeteringen te identificeren. 
De resultaten zullen worden verwerkt door het berekenen van gemiddelden, het analyseren van de spreiding van de data en het identificeren van eventuele verbeteringen ten opzichte van de nulmeting. Deze gegevens zullen cruciaal zijn voor het beoordelen van de effectiviteit van de mobiliteitstool en het informeren van verdere ontwikkelingen en aanpassingen. De POC zal worden uitgewerkt op een smartphone voor een aantal specifieke routes waarin minstens 1 openbaar vervoersmiddel (trein, bus, tram, ...) is geïntegreerd. Uiteindelijk zal een compilatie gemaakt worden van de beste opties en combinaties om verder uit te werken in de POC; 

\begin{enumerate}
    \item \textbf{Functies}: Een lijst van alle nodige functies in de app (location tracking, text-to-speech, voorleesfunctie, language models, visualisatietypes, Sclera pictogrammen)
    \item \textbf{Integraties}: Een omschrijving van mogelijke AR integraties (Google Maps, Mapbox, Koombea, AR City, Sygic)
    \item \textbf{Werkwijze en denkpatronen}: Een selectie van enkele mogelijke manieren waarmee interactie kan geïnitieerd worden inclusief AR (kleuren, geluiden, patronen, figuren, ...)
\end{enumerate}
