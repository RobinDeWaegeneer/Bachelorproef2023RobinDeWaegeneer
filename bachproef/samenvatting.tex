%%=============================================================================
%% Samenvatting
%%=============================================================================

% TODO: De "abstract" of samenvatting is een kernachtige (~ 1 blz. voor een
% thesis) synthese van het document.
%
% Een goede abstract biedt een kernachtig antwoord op volgende vragen:
%
% 1. Waarover gaat de bachelorproef?
% 2. Waarom heb je er over geschreven?
% 3. Hoe heb je het onderzoek uitgevoerd?
% 4. Wat waren de resultaten? Wat blijkt uit je onderzoek?
% 5. Wat betekenen je resultaten? Wat is de relevantie voor het werkveld?
%
% Daarom bestaat een abstract uit volgende componenten:
%
% - inleiding + kaderen thema
% - probleemstelling
% - (centrale) onderzoeksvraag
% - onderzoeksdoelstelling
% - methodologie
% - resultaten (beperk tot de belangrijkste, relevant voor de onderzoeksvraag)
% - conclusies, aanbevelingen, beperkingen
%
% LET OP! Een samenvatting is GEEN voorwoord!

%%---------- Nederlandse samenvatting -----------------------------------------
%
% TODO: Als je je bachelorproef in het Engels schrijft, moet je eerst een
% Nederlandse samenvatting invoegen. Haal daarvoor onderstaande code uit
% commentaar.
% Wie zijn bachelorproef in het Nederlands schrijft, kan dit negeren, de inhoud
% wordt niet in het document ingevoegd.

%\IfLanguageName{english}{%
%\selectlanguage{dutch}
%\chapter*{Samenvatting}
%%%\lipsum[1-4]
%\selectlanguage{english}
%}{}

%%---------- Samenvatting -----------------------------------------------------
% De samenvatting in de hoofdtaal van het document

\chapter*{\IfLanguageName{dutch}{Samenvatting}{Abstract}}

Dit eindwerk gaat over vergelijkende studie waarbij de best mogelijke navigatietool gekozen wordt voor mensen met een mentale beperking om zich te navigeren naar hun bestemming. De doelgroep ondervinden bij het gebruik van bestaande mobiliteitstools problemen. Problemen onder andere zoals: het onthouden van hun bestemming, de cognitieve belastingen van de applicaties, de meldingen van applicaties, \ldots Nu welke tool is de beste navigatietool voor mensen met een mentale beperking om zich te verplaatsen? Kan Augmented Reality (AR) hulp bieden bij het navigeren? Kort weg de doelstelling van dit onderzoek is de best mogelijke navigatietool te selecteren voor deze doelgroep en een eigen applicatie te ontwikkelen die aansluit bij de noden van de doelgroep. In de methodologie wordt er een diepere kijk gevormd op de fases en de verdere vorming van het onderzoek. In het onderdeel Proof of Concept ontwikkelen we een eigen applicatie en wordt er een concrete vergelijking gemaakt tussen Mapbox en Google Maps de geselecteerde navigatietools. Er kan geconcludeerd worden dat deze vrijwel dicht bij elkaar liggen ten opzichte van de requirements. Alhoewel er een klein verschil is tussen beiden en dat is dat Mapbox meer voor ontwikkelaars is en Google Maps een goede kant-en-klare oplossing biedt indien er geen technische kennis is voor het ontwikkelen is van een eigen applicatie. AR biedt ook een meerwaarde bij het navigeren en verkrijgen van duidelijke instructies om zich te verplaatsen van punt A naar punt B. Als laatste kan er gesteld worden dat aan de hand van Sclera pictogrammen en voor geprogrammeerde of ingestelde adressen kunnen helpen bij het onthouden van hun eindbestemming.
