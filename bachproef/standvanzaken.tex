\chapter{\IfLanguageName{dutch}{Stand van zaken}{State of the art}}
\label{ch:stand-van-zaken}

% Tip: Begin elk hoofdstuk met een paragraaf inleiding die beschrijft hoe
% dit hoofdstuk past binnen het geheel van de bachelorproef. Geef in het
% bijzonder aan wat de link is met het vorige en volgende hoofdstuk.

% Pas na deze inleidende paragraaf komt de eerste sectiehoofding.

%%Dit hoofdstuk bevat je literatuurstudie. De inhoud gaat verder op de inleiding, maar zal het onderwerp van de bachelorproef *diepgaand* uitspitten. De bedoeling is dat de lezer na lezing van dit hoofdstuk helemaal op de hoogte is van de huidige stand van zaken (state-of-the-art) in het onderzoeksdomein. Iemand die niet vertrouwd is met het onderwerp, weet nu voldoende om de rest van het verhaal te kunnen volgen, zonder dat die er nog %%andere informatie moet over opzoeken \autocite{Pollefliet2011}.

%%Je verwijst bij elke bewering die je doet, vakterm die je introduceert, enz.\ naar je bronnen. In \LaTeX{} kan dat met het commando \texttt{$\backslash${textcite\{\}}} of \texttt{$\backslash${autocite\{\}}}. Als argument van het commando geef je de ``sleutel'' van een ``record'' in een bibliografische databank in het Bib\LaTeX{}-formaat (een tekstbestand). Als je expliciet naar de auteur verwijst in de zin (narratieve referentie), gebruik je \texttt{$\backslash${}textcite\{\}}. Soms is de auteursnaam niet expliciet een onderdeel van de zin, dan gebruik je \texttt{$\backslash${}autocite\{\}} (referentie tussen haakjes). Dit gebruik je bv.~bij een citaat, of om in het bijschrift van een overgenomen afbeelding, broncode, tabel, enz. te %%verwijzen naar de bron. In de volgende paragraaf een voorbeeld van elk.

%%\textcite{Knuth1998} schreef een van de standaardwerken over sorteer- en zoekalgoritmen. Experten zijn het %%erover eens dat cloud computing een interessante opportuniteit vormen, zowel voor gebruikers als voor %%dienstverleners op vlak van informatietechnologie~\autocite{Creeger2009}.

%%Let er ook op: het \texttt{cite}-commando voor de punt, dus binnen de zin. Je verwijst meteen naar een bron in %%de eerste zin die erop gebaseerd is, dus niet pas op het einde van een paragraaf.
%%\lipsum[7-20]

%%Let er ook op: het \texttt{cite}-commando voor de punt, dus binnen de zin. Je verwijst meteen naar een bron in %%de eerste zin die erop gebaseerd is, dus niet pas op het einde van een paragraaf.

%%\lipsum[7-20]

%TODO Ik denk er ook net aan. Als je de tabellen in hoofdstuk 2 aanpast dat je ernaar verwijst in de tekst. Spreek ook over wat er in de tabel staat, niet zomaar de tabel geven.

Navigatie is de kunst van het plannen en volgen van een route om zich daarmee van de huidige positie naar de bestemming te verplaatsen. Het woord `navigatie' is afgeleid uit de Latijnse woorden \textit{navis}, dat schip betekent, en \textit{agere}, dat in deze context bewegen of sturen betekent. Bij navigatie gaat het erom dat je je eigen plaats bepaalt, de plaats bepaalt waar je heen gaat, en daarna de weg ernaartoe. Hoewel dit op zichzelf een evidentie lijkt, is het voor mensen met een mentale beperking niet eenvoudig deze taken uit te voeren aan de hand van de huidige beschikbare navigatietools.

De literatuurstudie naar de diverse navigatietechnologieën toegelicht in dit hoofdstuk zal een uitgebreide blik bieden op de huidige stand van de techniek binnen het domein van navigatiesystemen en hun evolutie over de tijd. Daarnaast wordt een duidelijk kader afgebakend over de betekenis van het begrip mentale beperking. Deze studie zal toelaten om in kaart te brengen hoe de technologische vooruitgang en toepassingen van navigatiesystemen zich kunnen richten op het verbeteren van de levenskwaliteit voor mensen met een beperking. Het is de essentiële schakel tussen de algemene aspecten omtrent de huidige navigatietechnologieën en de implementatie ervan voor mensen met een mentale beperking.

\section{Navigatie}
\label{sec:navigatie}

De oorsprong van onze huidige navigatiesystemen ligt in het NAVSTAR systeem dat het defensiedepartement van de Verenigde Staten samen met National Aeronautics and Space Administration (NASA) in 1967 ontwikkelde. De afkorting verwijst naar NAVigation Satellite Time And Ranging en laat toe om overal en continu te navigeren \autocite{Bowditch2002}. 

Vandaag staat het gekend als Global Positioning System (GPS) en is het geïntegreerd in vele toepassingen sinds de vrijgave ervan voor civiel gebruik in 1983. Oorspronkelijke gebruikers waren het leger, scheepvaart en landmeters, maar dankzij de technologische ontwikkeling (meer satellieten, goedkopere ontvangers\ldots) heeft de toepassing ook zijn weg gevonden naar de gewone consument, meerbepaald in hun auto's en mobiele telefoons. NASA publiceerde doorheen de tijd heel wat details omtrent de werking van GPS, zodat deze consumenten toepassingen mogelijk werden \autocite{Zaidman2008}. 

Naast GPS zijn er bovendien intussen nog drie nieuwe operationele satellietplaatsbepalingssystemen ontwikkeld door andere landen, nl. GLONASS (Rusland), BeiDou (China) en Galileo (Europese Unie), wat het belang van een dergelijk systeem benadrukt. 

Het huidige concept van locatiebepaling, zoals dat vandaag dikwijls gelinkt is aan een mobiele telefoon en diverse apps, is met andere woorden gebaseerd op de connectie met deze satellietsystemen. Hoe nauwkeuriger deze systemen worden, hoe accurater ook deze locatiebepaling zal zijn. 

Voor mensen met een visuele beperking volstaat de resolutie geboden door dergelijke GPS-systemen echter niet altijd. Daarom introduceerde \textcite{Katz2010} het NAVIG-systeem, dat Global Navigation Satellite System (GNSS) en snelle visuele herkenning combineert om visueel gehandicapte gebruikers te begeleiden in zowel bekende als onbekende omgevingen. Deze systemen benadrukken het belang van technologie voor het verbeteren van navigatie voor mensen met een beperking. Om hier iets aan te doen, ontwikkelde  
\textcite{Lakde2015} een navigatiehulpsysteem voor visueel gehandicapten, dat gebruik maakt van een combinatie van sensortechnologie en stembegeleiding. Dit systeem maakt gebruikers bewust van hun pad en eventuele obstakels. Deze ontwikkeling zet sterk in op visuele beperking en gaat uit van het idee dat degene die zich wil verplaatsen een goed begrip heeft van zijn huidige locatie en bestemming. Dankzij het verfijnen van de resolutie in positionering biedt de technologische progressie hier zeker oplossingen voor. 

Deze ontwikkeling komt echter niet tegemoet aan mensen met een beperking die in een onbekende omgeving belanden waarin ze een specifieke bestemming willen bereiken, die ze zelf niet voldoende duidelijk kunnen definiëren omwille van bijvoorbeeld een beperkte geletterdheid. In dat kader wordt in het volgende hoofdstuk omschreven wat exact bedoeld wordt als een mentale beperking. 

\section{Mentale beperking}
\label{sec:mentale-beperking}

% TODO: bron bij deze definitie en beweringen in deze paragraaf (\textbf{dit is de gevraagde referentie American Psychiatric Association, ed. (2022). \textit{Diagnostic and Statistical Manual of Mental Disorders, Fifth Edition, Text Revision (DSM-5-TR)}. Washington, DC, USA: American Psychiatric Publishing. \href{https://en.wikipedia.org/wiki/ISBN_(identifier)}{ISBN} \href{https://en.wikipedia.org/wiki/Special:BookSources/978-0-89042-575-6}{978-0-89042-575-6}}. )

Het begrip ``verstandelijke handicap'' of ``mentale beperking" wordt door het Vlaams Agentschap voor Personen met een Handicap (VAPH) gedefinieerd aan de hand van richtlijnen van de DSM-5, de American Association on Intellectual and Devel- opmental Disabilities (AAIDD), en het Classificerend Diagnostisch Protocol (CDP). Volgens de American Psychiatric Association (APA) betreft een verstandelijke handicap een stoornis die ontstaat tijdens de ontwikkeling en zowel beperkingen in intellectueel functioneren als aanpassingsproblemen op conceptueel, sociaal en praktisch niveau omvat.

De benadering van het VAPH richt zich daarenboven specifiek op het sociaal- ecologische perspectief, waarbij het functioneren van individuen wordt gezien als een interactie tussen de persoon en zijn omgeving. Het verkrijgen van ondersteuning bij dagelijkse activiteiten staat hierbij centraal, waarbij zowel persoonlijke als externe factoren van invloed zijn op het functioneren. Het is van belang dat bij het vaststellen van een verstandelijke handicap ook rekening wordt gehouden met de sterke punten van de betrokkene. 

Om de diagnose van een verstandelijke handicap te stellen, moeten drie criteria worden voldaan: het intelligentiecriterium, het criterium adaptief gedrag en het ontwikkelingscriterium. Het intelligentie criterium vereist een aantoonbare beperking in intellectueel functioneren, vaak vastgesteld met gestandaardiseerde intelligentietests. Het criterium adaptief gedrag omvat tekorten in aanpassingsgedrag op conceptueel, sociaal en praktisch gebied. Het ontwikkelingscriterium vereist dat deze beperkingen zich manifesteren vóór de leeftijd van 22 jaar. 

Personen met een verstandelijke handicap kunnen verder worden onderverdeeld op basis van de ernst ervan, wat voornamelijk voor on\-der\-zoeks- en rapportagedoeleinden wordt gedaan. Het gebruik van deze onderverdeling in de praktijk vereist echter zorgvuldige afweging, zoals beschreven in het diagnostisch protocol \autocite{VAPH}. Zoals hierboven aangegeven in de definitie van het VAPH is inzetten op de sterke punten van een persoon met een verstandelijke beperking belangrijk voor het functioneren.

In dit werk ligt de focus op navigatieoplossingen voor mensen met een mentale beperking die zowel minder goed intellectueel functioneren als moeilijker adaptief gedrag vertonen. Qua intellectuele beperking betekent dit dikwijls dat deze mensen beperkte lees- en schrijfvaardigheden hebben, maar net zoals iedereen hebben zij zelfstandige mobiliteitsbehoeften voor o.a. verplaatsingen van en naar huis, een maatwerkbedrijf of een vrijetijdsbesteding.

Lezen doen we allemaal zonder erbij stil te staan. Onze informatiemaatschappij gaat uit van dit basisprincipe. Alles is steeds meer digitaal, maar om te participeren en zelfstandig te functioneren, is lezen onmisbaar. De trein nemen, een busticket betalen, een plaats in de bioscoop boeken, info zoeken op het internet over een bepaalde locatie\ldots; al deze handelingen vereisen minimale geletterdheid. Deze doelgroep wordt dus snel geconfronteerd met uitdagingen bij het plannen en uitvoeren van verplaatsingen. 

Daarnaast kunnen moeilijkheden met adaptief gedrag, het probleemoplossend denken en omgaan met verandering, i.e. snel aanpassen aan een gewijzigde omgeving bijvoorbeeld, sterk belemmeren. Afhankelijk van hun type beperking kunnen deze noden bovendien sterk variëren. Bij autismespectrumstoornis (ASS) werden diverse aspecten omtrent tijdsbesef samengevat door \textcite{Degrieck2014}. Een grote gemeenschappelijke deler bij deze doelgroep is de nood aan eenduidige structuur, visuele ondersteuning en afwezigheid van onnodige prikkels die afleiden van het tijdsbesef. Zeker in de context van ASS is een advertentieluwe omgeving met enkel essentiële informatie een absolute noodzaak \autocite{Roeyers2014}. 

De website van Participate\footnote{\url{https://nl.participate-autisme.be}} bundelt hieromtrent heel wat ervaringsgerichte informatie via hun FAQ en enkele blogs. Mensen met een laag IQ kunnen bovendien slechts bepaalde letter- en cijfercombinaties memoriseren \autocite{DeGraaf2001, Tytgat2014}. \textcite{Tilborg2018} bestudeerde hiertoe welke tekenen van geletterdheid specifiek voor kinderen met intellectuele beperkingen nuttig zijn. \textcite{Uyttersprot2021} baseerde hierop haar werk `Lees je honger' waarin eten en drinken als basis aangewend worden voor functioneel en voorbereidend lezen voor kinderen in Type 2 buitengewoon onderwijs. Dit laatste werk geeft aan hoe vanuit een beperkte woordenschat, gebaseerd op voeding, specifieke letters kunnen aangeleerd worden via visuele herkenningspunten en zo enige leesvaardigheid kan worden bereikt. Daarnaast is de mogelijkheid om te kiezen voor een klassiek klokbeeld in plaats van het digitale beeld of een keuze voor bepaalde lettertypes een aspect dat voor deze mensen het verschil kan maken. Door de PoC af te stemmen op die beperkte leesvaardigheid kan de tool een oplossing vormen voor de specifieke doelgroep. 

Een bijkomende optie voor beide doelgroepen (beperkt intellectueel en adaptief vermogen) is het gebruik van standaard pictogrammen. Een algemeen erkend systeem zijn de zogenaamde Sclera picto's \footnote{\url{https://www.sclera.be/nl/vzw/geschiedenis}}. Ze vormen een sterke visuele ondersteuning en zijn een benchmark in deze context. Sclera is een vzw opgericht in 2008 met als missie om bij te dragen tot het realiseren van communicatieve en verstandelijke toegankelijkheid. De pictogrammen ontwikkeld door Bart Serrien werden gedurende enkele jaren uitgebreid en aangeboden als hulpmiddel aan diverse organisaties en doelgroepen. Door via hun werking te sensibiliseren, te informeren en advies te formuleren, vonden deze pictogrammen hun weg naar het onderwijs en vele professionele ondersteunende organisaties. Hun website omvat nog steeds een ruim overzicht aan tools met andere pictogramsystemen, software, apps, online tools en advies of vorming. Veel mensen met een mentale beperking, en hun begeleiders, kennen en gebruiken deze pictogrammen dagelijks. 

In de huidige state of the art is echter weinig tot geen informatie beschikbaar over het samen toepassen in mobiliteitstools van de verschillende hierboven omschreven technieken om zo optimaal bij te dragen tot begrijpbaarheid en functionaliteit. De noodzaak van accurate locatie gebaseerde informatie, het kiezen van een prikkelarme weergave, de mogelijkheid van het gebruik van pictogrammen voor het beschrijven van de bestemming en de integratie van spraakherkenning voor gebruikers die niet kunnen lezen of schrijven, is nieuw in deze context. In de volgende hoofdstukken wordt daartoe een afzonderlijke analyse gemaakt van de nieuwste beschikbare technologieën en hoe deze de beoogde implementatie van dergelijke functionaliteiten kunnen ondersteunen. 


\section{Augmented Reality}
\label{sec:augmented-reality}

Augmented Reality (AR) is een technologie die de gebruikerservaring verbetert door digitale informatie over objecten of plaatsen in de echte wereld te projecteren \autocite{Berryman2012}. Het verschilt van Virtual Reality (VR) doordat gebruikers nog steeds hun fysieke omgeving ervaren \autocite{Calo2015}. AR wordt op verschillende gebieden gebruikt, waaronder geneeskunde om artsen te ondersteunen bij complexe ingrepen, in marketing om gebruikerservaring te simuleren en in onderwijs om opleiding te concretiseren. AR wordt gedefinieerd als een realtime weergave van de fysieke wereld, versterkt door virtueel computer gegenereerde informatie \autocite{Carmigniani2011}. 

De integratie van AR kan met andere woorden extra mogelijkheden toevoegen aan de gebruikservaring van mensen met een verstandelijke beperking die een beperkter inlevings- en inbeeldingsvermogen hebben. Het visualisatie aspect kan een hulpmiddel zijn voor mensen met adaptatieproblemen die moeilijk om kunnen met gewijzigde omstandigheden. Google Maps maakt reeds gebruik van een statische visualisatie. Hieraan zou AR een `live' laag kunnen toevoegen door bijvoorbeeld het beeld aan te passen in functie van het seizoen.

Kortom, het is een veelzijdige technologie met potentieel positieve toepassingen, maar ze brengt ook beleidszorgen met zich mee die moeten worden aangepakt \autocite{Calo2015}. Hieronder zijn enkele belangrijke beleidsvormen omschreven die daarbij moeten overwogen moeten worden:  

\begin{itemize}
    \item \textbf{Privacybescherming}: Gezien AR apparaten in staat zijn om continue gegevens over de gebruikers en hun omgeving te verzamelen, is het cruciaal dat er sterke privacy richtlijnen en beleid worden geformuleerd en geïmplementeerd. Dit omvat beleid dat reguleert hoe deze gegevens worden verzameld, gebruikt, en gedeeld. De bescherming van persoonlijke gegevens moet voorop staan om misbruik en ongeautoriseerde toegang te voorkomen.
    \item \textbf{Gelijkheid en non-discriminatie}: Beleid moet ervoor zorgen dat AR technologieën toegankelijk zijn voor alle gebruikers, ongeacht socio-economische status, locatie of fysieke vermogens. Het beleid moet discriminatie voorkomen en inclusiviteit bevorderen. Specifiek deze pijler toont aan welke opportuniteiten dit biedt voor mensen met een mentale beperking.
    \item \textbf{Veiligheid en welzijn}: AR toepassingen kunnen potentiële veiligheidsrisico's met zich meebrengen, zoals afleiding en desoriëntatie. Beleidsmakers moeten richtlijnen ontwikkelen die de veiligheid van gebruikers garanderen, vooral in situaties waarbij de fysieke en digitale wereld overlappen.
    \item \textbf{Flexibiliteit en aanpasbaarheid}: Beleid moet flexibel genoeg zijn om aanpassingen toe te staan naarmate de technologie en maatschappelijke normen evolueren. Dit zorgt ervoor dat wetgeving relevant en effectief blijft in het licht van snelle technologische vooruitgang.
    \item \textbf{Ethische overwegingen}: Het ontwikkelen en implementeren van AR moet geleid worden door ethische overwegingen die rekening houden met de impact op individuen en de samenleving als geheel. Dit omvat het overwegen van de effecten van AR op sociale interacties, mentale gezondheid, en persoonlijke autonomie.
\end{itemize}

Deze beleidsvormen en zorgen zijn van essentieel belang om AR technologieën op een verantwoorde en effectieve manier in te zetten en tegelijk de rechten en het welzijn van individuen, in het bijzonder deze van mensen met een beperking te beschermen en tegelijk inclusie te bevorderen \autocite{Roesner2014}.

Tot slot is het nodig te begrijpen dat AR anders wordt ervaren door verschillende mensen en dus het is belangrijk diverse bevolkingsgroepen (leeftijd, cultuur\ldots) te raadplegen. Het systeem mag geen vooroordelen bevorderen op basis van verkeerd gevoede data. Het bouwen van dynamische systemen die flexibel zijn en kunnen bijgewerkt worden naarmate de technologie en cultuur veranderen zijn noodzakelijk. 

De algemene conclusie is dat AR systemen ontwikkeld moeten worden zodanig dat deze eenvoudig aanpasbaar naarmate de verandering in technologie en samenleving \autocite{Calo2015}.  De specifieke beleidszorgen moeten aangepakt moeten om de technologie toe te laten zich verder te integreren in diverse aspecten van het dagelijks leven.

\section{Artificiële Intelligentie}
\label{sec:artificiele-intelligentie}

Artificiële Intelligentie (AI) omvat het vermogen van machines of computersystemen om taken uit te voeren die traditioneel menselijke intelligentie vereisen, zoals logisch redeneren, leren, en problemen oplossen \autocite{Sabouret2020}. AI-systemen gebruiken complexe algoritmen en technologieën voor machinaal leren om cognitieve vaardigheden na te bootsen, waardoor ze in staat zijn autonoom te functioneren in diverse omgevingen. 

In de context van navigatie en routeplanning, stelt AI ons in staat om optimale routes te berekenen door realtime verkeersgegevens en geografische informatie te analyseren. Zo beschrijven \textcite{Hu2020} hoe een AI-gestuurd systeem logistieke voertuigen helpt hun route efficiënt aan te passen door continu GPS-gegevens te koppelen aan geplande paden, waardoor snelle reacties op verkeerswijzigingen mogelijk zijn. Door gebruik te maken van geavanceerde datastructuren, kunnen AI-systemen locatie-informatie koppelen aan routepatronen, waardoor het mogelijk wordt knelpunten te identificeren en routes aan te passen aan veranderende omstandigheden zoals verkeersopstoppingen of wegwerkzaamheden \autocite{Soni2023a,Ruta2010}. 

Het belang van realtime gegevens is cruciaal in deze systemen, omdat deze de AI in staat stellen snel en efficiënt te reageren op dynamische veranderingen in de omgeving \autocite{Ciravegna2018}. Deze AI-gestuurde aanpak heeft het potentieel om de efficiëntie en nauwkeurigheid van navigatiesystemen aanzienlijk te verbeteren, waardoor gebruikers tijd kunnen besparen en hun reiservaring kunnen optimaliseren. Door voortdurend de nieuwste technologieën en methodieken te integreren, kunnen we verwachten dat AI een steeds belangrijkere rol zal spelen in de evolutie van intelligente transportsystemen.

Voor mensen met een mentale beperking kan dit concreet betekenen dat het AI-systeem ontbrekende informatie kan aanvullen of voorstellen op basis van gekende patronen (dagelijkse verplaatsing, eerder bezochte locaties, specifieke referentie personen of instanties). Wanneer de persoon met mentale beperking slechts over een deel van de informatie beschikt of deze enkel gedeeltelijk kan reproduceren, vult het AI-systeem op een logische wijze verder aan. AI zal in de context van de doelgroep een duidelijke meerwaarde kunnen bieden in het dagelijks leven.

\section{Hedendaagse navigatiemethodes}
\label{sec:hedendaagse-navigatiemethodes}

% je kan hier bespreken wat je gaat doen in deze sectie, waar je naar gaat kijken, of je bepaalde tools al uitsluit (en waarom), enz.

Navigatietools zijn essentieel in onze steeds mobieler wordende samenleving. Diverse methoden zijn nuttig voor het plannen van routes en het bieden van navigatie-instructies. Dit hoofdstuk gaat dieper in op een aantal hiervan met bijzondere aandacht voor hun bruikbaarheid voor personen met een mentale beperking.  Ze stellen allen gebruikers in staat efficiënt en veilig van punt A naar punt B te reizen, maar wat vooral belangrijk is voor personen met een mentale beperking zijn de opties die ze bieden voor mogelijks extra ondersteuning tijdens hun verplaatsingen. De voor- en nadelen van vier specifieke tools worden kort opgesomd, nl. Google Maps, Mapbox, Sygic, en De Lijn en NMBS.

\subsection{Google Maps}
\label{sec:google-maps}
Google Maps\footnote{\url{https://www.google.com/maps}}, gelanceerd in 2005 door Google Inc., heeft zich ontwikkeld van een eenvoudige kaartdienst tot een van de meest uitgebreide en multifunctionele navigatietools wereldwijd. Het biedt gedetailleerde geografische informatie door het combineren van satellietbeelden, luchtfotografie, en straatkaarten, en faciliteert routeplanning voor diverse vervoersmiddelen, waaronder auto's, openbaar vervoer, fietsen, en lopen. 

Volgens een onderzoek van \textcite{Mehta2019}, biedt Google Maps uitgebreide analytische mogelijkheden door zowel historische als real-time data te combineren. Deze mogelijkheden stellen gebruikers in staat om uitgebreide analyses en gewenste operaties uit te voeren, zoals het plannen van routes en het ontvangen van verkeersupdates in real-time. Deze functionaliteiten onderstrepen de rol van Google Maps als een cruciaal instrument in stedelijke planning en mobiliteitsbeheer \autocite{Mehta2019}. In een studie over gebruikersgedrag en data-extractie door \textcite{Sardianos2018} wordt gedemonstreerd hoe Google Maps kan worden ingezet om gebruikersgewoonten te extraheren uit de geschiedenisarchieven. Deze gegevens zijn van onschatbare waarde voor stedelijke planners en verkeersbeheerders om verkeersstromen te begrijpen en te optimaliseren, wat leidt tot betere besluitvorming en serviceverbetering \autocite{Sardianos2018}. Een aantal voor- en nadelen die bijzondere aandacht vragen in ons onderzoek worden hier opgelijst:

\subsubsection*{Voordelen}
\begin{itemize}
    \item Toegankelijk en gebruiksvriendelijk met een brede reeks functies.
    \item Wereldwijde dekking met nauwkeurige kaartgegevens.
    \item Real-time verkeersupdates en routealternatieven.
\end{itemize}
\subsubsection*{Nadelen}
\begin{enumerate}
    \item Privacyzorgen: zorgen over dataverzameling en -gebruik.
    \item Complexiteit in stedelijke gebieden: Kan soms verwarrend zijn.
    \item Afhankelijk van internet: vereist een actieve internetverbinding.
\end{enumerate}

\subsection{Mapbox}
Mapbox\footnote{\url{https://www.mapbox.com}} werd opgericht in 2010 en onderscheidt zich door het bieden van aanpasbare kaartoplossingen die vooral gericht zijn op ontwikkelaars en bedrijven. Mapbox stelt gebruikers in staat om unieke en krachtige kaarten te creëren voor websites en mobiele applicaties door gebruik te maken van hun uitgebreide datasets en API's. De dienst biedt uitgebreide stylingsopties en tools voor gegevensvisualisatie, waarmee bedrijven kaarten kunnen aanpassen aan hun specifieke branding en gebruikerservaringen. Het platform ondersteunt ook geavanceerde locatiefuncties zoals turn-by-turn navigatie, geospatiale analyses, en integratie van live locatiedata, wat het een populaire keuze maakt voor applicaties die dynamische locatiegebaseerde diensten vereisen. 

Een studie door \textcite{Neene2017} illustreert de ontwikkeling van een mobiele GIS-eigendomskaarttoepassing met behulp van Mapbox-technologieën. Hierbij werd gebruik gemaakt van satellietbeelden en vectorisatietools om gedetailleerde kaarten te produceren voor mobiele cloud computing toepassingen, wat de flexibiliteit van Mapbox in verschillende technologische omgevingen benadrukt. De integratie van Mapbox met andere platforms zoals Google Maps verbetert de functionaliteit van geografische informatiesystemen. Deze integratie stelt ontwikkelaars in staat om uitgebreide features te benutten en op maat gemaakte oplossingen te bieden die aansluiten bij specifieke zakelijke behoeften \autocite{Hidayatulloh2023}. Een aantal voor- en nadelen die bijzondere aandacht vragen in ons onderzoek worden hier opgelijst:

\subsubsection*{Voordelen}
\begin{itemize}
    \item Biedt uitgebreide aanpassingsmogelijkheden voor kaartontwerp.
    \item Sterke API-ondersteuning voor ontwikkelaars.
    \item Flexibele styling en uitgebreide ontwerp opties.
\end{itemize}
\subsubsection*{Nadelen}
\begin{enumerate}
    \item Technische complexiteit: vereist technische kennis voor implementatie.
    \item Kosten kunnen oplopen bij hoger gebruik.
    \item Aanpassing noodzakelijk voor unieke stijlen.
\end{enumerate}

\subsection{Sygic}
Sygic\footnote{\url{https://www.sygic.com}} werd opgericht in 2004 en is een van de eerste GPS-navigatie-apps die downloadbare kaarten biedt voor offline gebruik. Het is vooral gericht op autonavigatie en biedt uitgebreide navigatiediensten zoals stemgeleide navigatie, snelheidslimietwaarschuwingen en rijstrookaanwijzingen. De app maakt gebruik van hoogwaardige TomTom-kaarten en biedt geavanceerde navigatiefuncties zoals real-time verkeersinformatie, politieradarwaarschuwingen, en een head-up display (HUD) modus die navigatie-instructies direct op de voorruit van de auto projecteert. 

Een studie door \textcite{Wnorowski2018} illustreert hoe Sygic AR-technologie integreert om rijervaringen te verbeteren. Deze technologische vooruitgang stelt gebruikers in staat om via hun smartphonescherm een overlay van navigatie en wegdetails te zien, wat vooral nuttig is in complexe verkeerssituaties en onbekende gebieden. In een onderzoek waar verschillende tools vergeleken werden, werd Sygic gepresenteerd als een efficiënte tool voor het selecteren van de snelste routes in stedelijke omgevingen. Sygic's hoge gebruikspercentage en betrouwbaarheid in het vinden van optimale paden in de context van een slimme stad is een duidelijk pluspunt \autocite{Putra2021}. Een aantal voor- en nadelen die bijzondere aandacht vragen in ons onderzoek worden hier opgelijst:

\subsubsection*{Voordelen}
\begin{itemize}
    \item Offline kaarten beschikbaar wereldwijd.
    \item Eenvoudige en intuïtieve bediening.
    \item Uitgebreide veiligheidsfeatures.
\end{itemize}
\subsubsection*{Nadelen}
\begin{enumerate}
    \item Premium functies vereisen een abonnement.
    \item Interface kan overweldigend zijn voor nieuwe gebruikers.
    \item Beperkte klantenservice ervaringen.
\end{enumerate}

\subsection{De Lijn en NMBS}
De Lijn\footnote{\url{https://www.delijn.be/nl/}} is de openbare vervoersdienst in Vlaanderen, België, en biedt bus- en tramdiensten. De bijhorende app helpt reizigers met het plannen van hun reizen binnen het netwerk van De Lijn. Het biedt dienstregelingen, routeplanning, en real-time updates over dienststatus en -wijzigingen. De app is ontworpen om de toegankelijkheid tot openbaar vervoer te vergroten en biedt functies zoals reisplanning van deur tot deur, tariefinformatie, en locatie van haltes en stations. Enige kennis van het Lijn netwerk en de verschillende types diensten (gebruik van bepaalde symboliek) is soms vereist om de informatie helder te verwerken.

Nationale Maatschappij der Belgische Spoorwegen (NMBS)\footnote{\url{https://www.belgiantrain.be/nl}} is de nationale spoorwegoperator in België en biedt uitgebreide treindiensten aan doorheen het land. De NMBS-app faciliteert reisplanning voor treinreizigers en biedt functies zoals dienstregelingen, ticketaankoop, reserveringen, real-time treinstatus en stationinformatie. Deze app is essentieel voor dagelijkse pendelaars en toeristen in België, omdat het helpt bij het naadloos navigeren van het uitgebreide spoorwegnet van het land. Waar geen treinverbinding beschikbaar is, stelt de app bovendien ook alternatieve andere openbaar vervoersoplossingen voor. Opnieuw geldt dat het gebruik van specifieke symboliek gelinkt aan het type vervoersmiddel de informatie minder helder maakt.

Beide tools zijn overduidelijk gericht op ervaren gebruikers van het openbaar vervoersnetwerk. Een aantal voor- en nadelen die bijzondere aandacht vragen in ons onderzoek worden hier opgelijst:

\subsubsection*{Voordelen}
\begin{itemize}
    \item Uitgebreide integratie van openbaar vervoer informatie.
    \item Real-time updates over dienstregelingen en vertragingen.
    \item Handig voor dagelijkse pendelaars.
\end{itemize}
\subsubsection*{Nadelen}
\begin{itemize}
    \item Interface soms complex en niet intuïtief.
    \item Soms onnauwkeurige real-time data.
    \item Minder functionaliteit buiten stedelijke centra.
\end{itemize}

\section{Requirements van de gezochte navigatietool}
\label{sec:requirements-van-de-gezochte-navigatietool}

Het ontwikkelen van een geavanceerde navigatietool vereist een zorgvuldige afweging van verschillende technische en gebruikersgerichte factoren. Deze sectie beschrijft de vereisten die van cruciaal belang zijn voor de functionaliteit en het succes van de navigatietool. Een eerste stap hierbij is de MoSCoW-methode om de vereisten te prioriteren en verduidelijken vervolgens specifieke functionele en niet-functionele eisen die de tool moet vervullen. Door deze systematische aanpak kan een tool ontworpen worden die niet alleen technisch haalbaar is, maar ook optimaal aansluit bij de behoeften van de gebruikers.

\subsection{Wat is MoSCoW?}

De MoSCoW-methode is een prioriteringsframework gebruikt in projectbeheer om projectvereisten te categoriseren in vier categorieën: Must have, Should have, Could have, en Won't have. Deze methode bevordert een efficiënte middelentoewijzing en zorgt ervoor dat essentiële taken prioriteit krijgen. Dit is bijzonder nuttig binnen Agile projectbeheer voor het beheer van de projectomvang en het verbeteren van de productiviteit door duidelijk niet-essentiële elementen die niet worden aangepakt te definiëren \autocite{Brush2023}.

\subsection{Functionele requirements}
\label{sec:functionele-requirements}

In deze sectie worden de functionele opgesteld aan de hand van de MoSCoW-methode. Door deze methode toe te passen, wordt duidelijk welke functionaliteiten nodig zijn voor de applicatie, welke minder cruciaal zijn, welke optioneel zijn en welke buiten de scope liggen van de applicatie. Hierdoor ligt de aandacht op de kritieke aspecten van de van de tool.

\subsubsection{Must have}
\begin{itemize}
    \item \textbf{Algemene bruikbaarheid}: Het systeem moet toegankelijk zijn voor alle gebruikers, ongeacht hun specifieke beperkingen.
    \item \textbf{Platform}: Moet ontwikkeld worden in React Native om compatibiliteit met diverse mobiele platformen te garanderen.
    \item \textbf{Intuïtiviteit}: Het ontwerp moet duidelijk en eenvoudig zijn, met minimale complexiteit.
    \item \textbf{Platformonafhankelijkheid}: Het systeem moet functioneren onafhankelijk van het mobiele besturingssysteem.
    \item \textbf{Visuele aanwijzingen}: Duidelijke visuele aanwijzingen zoals symbolen, pijlen of kleurcoderingen moeten de gebruiker door de route leiden.
    \item \textbf{Auditieve instructies}: Moet gesproken aanwijzingen of geluidssignalen bieden voor gebruikers die visuele informatie moeilijk kunnen verwerken.
    \item \textbf{Real-time updates}: Continue updates over de locatie van de gebruiker en aanpassingen aan veranderende omstandigheden zijn essentieel.
    \item \textbf{Toegankelijkheidsfuncties}: Ondersteuning voor schermlezers, aanpasbare lettertypen en kleuren, en vergrotingsmogelijkheden.
    \item \textbf{Connectiviteit}: Het systeem moet functioneel blijven bij kleine onderbrekingen in de internetverbinding.
\end{itemize}

\subsubsection{Should have}
\begin{itemize}
    \item \textbf{Aanpasbaarheid}: Het systeem moet aanpasbaar zijn aan de individuele behoeften van de gebruiker, zoals voorkeursroutes en het vermijden van bepaalde obstakels.
    \item \textbf{Offline functionaliteit}: Het systeem moet bepaalde functies offline kunnen uitvoeren.
\end{itemize}

\subsubsection{Could have}
\begin{itemize}
    \item \textbf{Veiligheidsfuncties}: Ingebouwde waarschuwingen voor gevaarlijke gebieden en suggesties voor veiligere routes.
\end{itemize}

\subsubsection{Won't have}
\begin{itemize}
    \item \textbf{Visual noise}: Het systeem mag geen visuele of auditieve verstoringen veroorzaken die de gebruiker kunnen afleiden.
    \item \textbf{Speciale hardware}: Het systeem vereist geen speciale hardware voor augmented reality (AR) functies.
\end{itemize}

\subsection{Niet-functionele requirements}
\label{sec:niet-functionele-requirements}

Deze sectie behandelt de niet-functionele requirements. Deze requirements omvatten de kwalitatieve aspecten die essentieel zijn voor het creëren van een gebruiksvriendelijke navigatietool die een aangename ervaring biedt aan de gebruikers uit de specifieke doelgroep van mensen met een mentale beperking. We richten ons hier op aspecten zoals de aanpasbaarheid, toegankelijkheid en algemene bruikbaarheid van de navigatiemethode.

\subsubsection{Must have}
\begin{itemize}
    \item Het navigatieproces moet snel zijn, om tijdverlies en onzekerheid te minimaliseren.
    \item De informatie die de gebruiker moet onthouden moet beperkt zijn, om het risico op fouten te verlagen.
    \item Het aantal stappen in het navigatieproces moet tot een minimum beperkt blijven.
\end{itemize}

\subsubsection{Should have}
\begin{itemize}
    \item De navigatiemethodes moeten gebruiksvriendelijk zijn voor mensen met dyslexie. Er moet rekening gehouden worden met de lees- en onthoudmoeilijkheden die zij kunnen ervaren.
\end{itemize}

\subsubsection{Could have}
\begin{itemize}
    \item Het zou voordelig zijn als de navigatiemethodes kostenefficiënt zijn, zodat onnodige kosten vermeden worden, wat de dienst toegankelijker maakt voor de eindgebruiker en de aanbieder.
\end{itemize}

\subsubsection{Won't have}

Er zijn geen won't have requirements.

\section{Het bepalen van de geschikte navigatiemethode}
\label{sec:bepalen-geschikte-navigatiemethode}

In deze sectie worden de verschillende navigatiemethodes geëvalueerd om de meest geschikte methode te identificeren voor mensen met een mentale beperking. Het doel is om een methode te vinden die de zelfstandigheid van de gebruiker maximaliseert en tegelijkertijd de cognitieve belasting minimaliseert.

\begin{table}[ht]
    \centering
    \begin{tabular}{|l|c|c|c|}
        \hline
        \textbf{Vereisten} & \textbf{Mapbox} & \textbf{Google Maps} \\ \hline
        Algemene Bruikbaarheid (M1) & X & X \\
        Platform (M2) & X & X \\
        Intuïtiviteit (M3) & X & X \\
        Platformonafhankelijkheid (M4) & X & X \\
        Visuele Aanwijzingen (M5) & X & X \\
        Auditieve Instructies (M6) & X & X \\
        Real-time Updates (M7) & X & X \\
        Toegankelijkheidsfuncties (M8) & X & X \\
        Connectiviteit (M9) & X & X \\
        Aanpasbaarheid (S1) & X & X \\
        Offline Functionaliteit (S2) & X & X \\
        Veiligheidsfuncties (C1) & X & X \\
        Visual noise (W1) & X & X \\
        Speciale Hardware (W2) & X & X \\
        Snelheid (M1) & X & X \\
        Beperkte Geheugenbelasting (M2) & X & X \\
        Minimalisatie van Navigatiestappen (M3) & X & X \\
        Gebruiksvriendelijkheid voor Dyslexie (S1) & X & X \\
        Kosten-efficiëntie (C1) & X & X \\ \hline
    \end{tabular}
    \caption{Vereisten voor navigatietools}
    \label{tab:vereisten voor navigatietools}
\end{table}

\subsection{Criteria voor Evaluatie}
\begin{itemize}
    \item \textbf{Gebruiksgemak:} Hoe intuïtief en eenvoudig is de interface voor de gebruiker?
    \item \textbf{Toegankelijkheid:} Biedt de methode ondersteuning voor toegankelijkheidsfuncties zoals spraakherkenning, aanpasbare tekstgrootte, en visuele of auditieve hulpmiddelen?
    \item \textbf{Reactievermogen:} Hoe goed kan de methode omgaan met veranderingen in de omgeving of bij onvoorziene omstandigheden?
    \item \textbf{Privacy:} Hoe gaat de methode om met gebruikersgegevens en privacybescherming?
\end{itemize}

\subsection{Evaluatie van Methodes}
Er zijn verschillende bestaande technologieën beschouwd, zoals Google Maps, Mapbox en gespecialiseerde apps zoals Sygic, en beoordeeld op de bovenstaande criteria. De methodes worden ook getest in een reële omgeving om hun effectiviteit in praktijksituaties te bepalen. Er wordt eerst een nulmeting uitgevoerd.

Zoals weergegeven in de tabel \ref{tab:vereisten voor navigatietools}, bieden Mapbox en Google Maps vrijwel hetzelfde aan qua functionaliteiten. Ze voldoen aan alle gestelde vereisten die nodig zijn voor een efficiënte applicatie.

\subsection{Aanbevolen Methode}
Na grondige evaluatie en praktijktests wordt de meest geschikte navigatiemethode aanbevolen. Deze aanbeveling is gebaseerd op de mate waarin de methode voldoet aan de criteria en de feedback van testgebruikers.

\subsection{Implementatievoorstel}
Er wordt een gedetailleerd plan opgesteld voor de implementatie van de aanbevolen navigatiemethode, inclusief stappen voor verdere ontwikkeling en integratie met bestaande systemen om de toegankelijkheid voor de doelgroep te vergroten.

\section{Technologieën}
\label{sec:technologieën}

In dit gedeelte wordt een overzicht gegeven van verschillende technologieën die van belang zijn voor het ontwikkelen van de applicatie. Technologieën evolueren voortdurend en spelen een cruciale rol in het bepalen van de functionaliteit, prestaties en gebruiksvriendelijkheid van softwaretoepassingen. Het selecteren van de juiste technologieën is essentieel voor het bereiken van de gewenste resultaten en het voldoen aan de behoeften van gebruikers voor navigatie. Daarom worden hier enkele van de belangrijkste technologieën besproken die worden gebruikt in het ontwikkelingsproces van de applicatie of PoC, waaronder JavaScript en React Native. Deze technologieën bieden krachtige tools en frameworks die ontwikkelaars in staat stellen om dynamische en responsieve gebruikersinterfaces te bouwen, zowel voor webapplicaties als voor mobiele apps.

\subsection{Javascript}
\label{sec:javascript}

JavaScript is een scripttaal waarmee je statische webapplicaties kan verbeteren met dynamische, gepersonaliseerde en interactieve inhoud. Dit verbetert de ervaring van bezoekers op uw site en maakt het waarschijnlijker dat ze opnieuw langskomen. Je hebt de vaste flikkerende uitklapmenu's, bewegende tekst en veranderende inhoud die nu wijdverspreid zijn op websites. Ze worden mogelijk gemaakt door JavaScript. JavaScript wordt ondersteund door alle grote browsers en is de taal bij uitstek op het web. Het kan zelfs worden gebruikt buiten webapplicaties, bijvoorbeeld om administratieve taken te automatiseren \autocite{Wilton2004}.

\subsection{React Native}
\label{sec:react native}

React is een bibliotheek waarmee ontwikkelaars gebruikersinterfaces (UI's) kunnen bouwen als een boom van kleine stukjes die componenten worden genoemd. Een component is een mix van HTML en JavaScript die alle logica bevat die nodig is om een klein deel van een grotere UI weer te geven. Elk van deze componenten kan worden opgebouwd tot opeenvolgende complexe onderdelen van een app \autocite{Baer2018}. React Native is een populair open-source framework, dat is ontwikkeld door Facebook. Ontwikkelaars kunnen hiermee mobiele applicaties bouwen aan de hand van JavaScript. React Native applicaties gebruiken de kracht van React, een JavaScript bibliotheek, voor het bouwen van gebruikersinterfaces met mobiele componenten. Door deze componenten kunnen ontwikkelaars efficiënt applicaties maken die werken voor zowel iOS als Android apparaten \autocite{Vinnik2021}. De keuze voor React Native als framework voor het ontwikkelen van mobiele applicaties is gebaseerd op verschillende factoren. Allereerst biedt React Native een efficiënte manier om cross-platform mobiele apps te bouwen met behulp van een enkele codebase, wat resulteert in een kosten- en tijdsbesparing. Bovendien maakt het gebruik van JavaScript het eenvoudiger voor ontwikkelaars om snel te leren en beginnen met het bouwen van mobiele applicaties. Tot slot heeft React Native een actieve community en een groot aantal beschikbare bibliotheken en plugins, waardoor ontwikkelaars toegang hebben tot een breed scala aan functionaliteiten en mogelijkheden voor het uitbreiden van hun applicaties.
