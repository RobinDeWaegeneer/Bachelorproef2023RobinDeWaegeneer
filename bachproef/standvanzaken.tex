\chapter{\IfLanguageName{dutch}{Stand van zaken}{State of the art}}%
\label{ch:stand-van-zaken}
% Tip: Begin elk hoofdstuk met een paragraaf inleiding die beschrijft hoe
% dit hoofdstuk past binnen het geheel van de bachelorproef. Geef in het
% bijzonder aan wat de link is met het vorige en volgende hoofdstuk.

% Pas na deze inleidende paragraaf komt de eerste sectiehoofding.

%%Dit hoofdstuk bevat je literatuurstudie. De inhoud gaat verder op de inleiding, maar zal het onderwerp van de bachelorproef *diepgaand* uitspitten. De bedoeling is dat de lezer na lezing van dit hoofdstuk helemaal op de hoogte is van de huidige stand van zaken (state-of-the-art) in het onderzoeksdomein. Iemand die niet vertrouwd is met het onderwerp, weet nu voldoende om de rest van het verhaal te kunnen volgen, zonder dat die er nog %%andere informatie moet over opzoeken \autocite{Pollefliet2011}.

%%Je verwijst bij elke bewering die je doet, vakterm die je introduceert, enz.\ naar je bronnen. In \LaTeX{} kan dat met het commando \texttt{$\backslash${textcite\{\}}} of \texttt{$\backslash${autocite\{\}}}. Als argument van het commando geef je de ``sleutel'' van een ``record'' in een bibliografische databank in het Bib\LaTeX{}-formaat (een tekstbestand). Als je expliciet naar de auteur verwijst in de zin (narratieve referentie), gebruik je \texttt{$\backslash${}textcite\{\}}. Soms is de auteursnaam niet expliciet een onderdeel van de zin, dan gebruik je \texttt{$\backslash${}autocite\{\}} (referentie tussen haakjes). Dit gebruik je bv.~bij een citaat, of om in het bijschrift van een overgenomen afbeelding, broncode, tabel, enz. te %%verwijzen naar de bron. In de volgende paragraaf een voorbeeld van elk.

%%\textcite{Knuth1998} schreef een van de standaardwerken over sorteer- en zoekalgoritmen. Experten zijn het %%erover eens dat cloud computing een interessante opportuniteit vormen, zowel voor gebruikers als voor %%dienstverleners op vlak van informatietechnologie~\autocite{Creeger2009}.

%%Let er ook op: het \texttt{cite}-commando voor de punt, dus binnen de zin. Je verwijst meteen naar een bron in %%de eerste zin die erop gebaseerd is, dus niet pas op het einde van een paragraaf.
%%\lipsum[7-20]

%%Let er ook op: het \texttt{cite}-commando voor de punt, dus binnen de zin. Je verwijst meteen naar een bron in %%de eerste zin die erop gebaseerd is, dus niet pas op het einde van een paragraaf.

%%\lipsum[7-20]

%TODO: hier moet nog een inleidende paragraaf komen, niet meteen beginnen met een sectie (zie ook de tip bovenaan) Ik denk wel dat je sectie 2.1 kan schrappen en kan gebruiken als inleiding voor dit hoofdstuk, hiermee kan je ook meteen de link leggen met sectie 2.2 (Mentale beperking)
%TODO: zorg ook voor een duidelijke overgang tussen elke 

Dit hoofdstuk verdiept de inleiding van de bachelorproef door de literatuurstudie naar navigatietechnologieën uit te voeren. Het biedt een uitgebreide blik op de huidige stand van de techniek binnen het domein van navigatiesystemen en hun evolutie over de tijd. Door deze literatuurstudie zal de lezer volledig op de hoogte zijn van zowel de technologische vooruitgang als de toepassingen ervan, specifiek gericht op het verbeteren van de levenskwaliteit voor mensen met een beperking. Dit hoofdstuk vormt een essentiële schakel tussen de algemene inleiding van de thesis en de gedetailleerde bespreking van navigatietechnologieën voor mensen met een mentale beperking, die in de volgende secties aan bod komt.
\section{Navigatie}
\label{sec:navigatie}

Navigatie is de kunst van het plannen en volgen van een route om zich daarmee van de huidige positie naar de bestemming te verplaatsen. Het woord ’navigatie’ is afgeleid uit de Latijnse woorden navis, dat schip betekent, en agere, dat in deze context bewegen of sturen betekent. Bij navigatie gaat het erom dat je je eigen plaats bepaalt en de plaats bepaalt waar je heen gaat en daarna de weg ernaartoe. 
In 1967 ontwikkelde het defensiedepartement van de Verenigde Staten samen met NASA (National Aeronautics and Space Administration) het NAVSTAR-systeem, verwijzend naar Navigation Satellite Time And Ranging, dat toelaat om overal en continu te navigeren \autocite{Bowditch2002}. 


Vandaag staat het gekend als GPS (Global Positioning System) en is het geïntegreerd in vele toepassingen sinds de vrijgave ervan voor civiel gebruik in 1983. Oorspronkelijke gebruikers waren het leger, scheepvaart en landmeters, maar dankzij de technologische ontwikkeling (meer satellieten, goedkopere ontvangers, …) heeft de toepassing ook zijn weg gevonden naar de gewone consument, meerbepaald in hun auto’s en mobiele telefoons. NASA publiceerde doorheen de tijd heel wat details omtrent de performance van GPS, zodat deze consumenttoepassingen mogelijk werden \autocite{Zaidman2008}.

Naast GPS zijn er bovendien intussen nog 3 nieuwe operationele satelliet plaatsbepalingssystemen ontwikkeld door andere landen, nl. GLONASS (Rusland), BeiDou (China) en Galileo (Europese Unie), wat het belang van een dergelijk systeem benadrukt.

Het huidige concept van locatiebepaling zoals dat vandaag dikwijls gelinkt is aan een mobiele telefoon en diverse apps is met andere woorden gebaseerd op de connectie met deze satellietsystemen. Hoe nauwkeuriger deze systemen worden, hoe accurater ook deze locatiebepaling zal zijn.

Voor mensen met een visuele beperking volstaat de resolutie geboden door dergelijke GPS systemen echter niet altijd. Daarom introduceerde \textcite{Katz2010} het NAVIG-systeem, dat GNSS (Global Navigation Satellite System) en snelle visuele herkenning combineert om visueel gehandicapte gebruikers te begeleiden in zowel bekende als onbekende omgevingen. Deze systemen benadrukken het belang van technologie voor het verbeteren van navigatie voor mensen met een beperking. Om hier iets aan te doen, ontwikkelde  
\textcite{Lakde2015} een navigatiehulpsysteem voor visueel gehandicapten, dat gebruik maakt van een combinatie van sensortechnologie en stembegeleiding. Dit systeem maakt gebruikers bewust van hun pad en eventuele obstakels. Deze ontwikkeling zet sterk in op visuele beperking en gaat uit van het idee dat degene die zich wil verplaatsen een goed begrip heeft van zijn huidige locatie en bestemming. Dankzij het verfijnen van de resolutie in positionering biedt de technologische progressie hier zeker oplossingen toe. Deze ontwikkeling komt echter niet tegemoet aan mensen met een beperking die in een onbekende omgeving belanden waarin ze een specifieke bestemming willen bereiken, die ze zelf niet voldoende duidelijk kunnen definiëren omwille van bijvoorbeeld een beperkte geletterdheid.

\section{Mentale beperking}
\label{sec:mentale-beperking}

Het begrip ”verstandelijke handicap” of mentale beperking wordt door het Vlaams Agentschap voor Personen met een Handicap (VAPH) gedefinieerd aan de hand van richtlijnen van de DSM-5, de American Association on Intellectual and Devel- opmental Disabilities (AAIDD), en het Classificerend Diagnostisch Protocol (CDP). Volgens de American Psychiatric Association (APA) betreft een verstandelijke handicap een stoornis die ontstaat tijdens de ontwikkeling en zowel beperkingen in intellectueel functioneren als aanpassingsproblemen op conceptueel, sociaal en praktisch niveau omvat. De benadering van het VAPH richt zich op het sociaal- ecologische perspectief, waarbij het functioneren van individuen wordt gezien als een interactie tussen de persoon en zijn omgeving. Het verkrijgen van ondersteuning bij dagelijkse activiteiten staat hierbij centraal, waarbij zowel persoonlijke als externe factoren van invloed zijn op het functioneren. Het is van belang dat bij het vaststellen van een verstandelijke handicap ook rekening wordt gehouden met de sterke punten van de betrokkene. Om de diagnose van een verstandelijke handicap te stellen, moeten drie criteria worden voldaan: het intelligentiecriterium, het criterium adaptief gedrag en het ontwikkelingscriterium. Het intelligentie criterium vereist een aantoonbare beperking in intellectueel functioneren, vaak vast- gesteld met gestandaardiseerde intelligentietests. Het criterium adaptief gedrag omvat tekorten in aanpassingsgedrag op conceptueel, sociaal en praktisch gebied. Het ontwikkelingscriterium vereist dat deze beperkingen zich manifesteren vóór de leeftijd van 22 jaar. Personen met een verstandelijke handicap kunnen verder worden onderverdeeld op basis van de ernst ervan, wat voornamelijk voor onderzoeks- en rapportagedoeleinden wordt gedaan. Het gebruik van deze onderverdeling in de praktijk vereist echter zorgvuldige afweging, zoals beschreven in het diagnostisch protocol \autocite{VAPH}. Zoals aangegeven in de definitie van het VAPH is inzetten op de sterke punten van een persoon met een verstandelijke beperking belangrijk voor het functioneren. In dit werk ligt de focus op navigatie oplossingen voor mensen met een mentale beperking die zowel minder goed intellectueel functioneren als moeilijker adaptief gedrag vertonen. Dikwijls betekent het dat deze mensen beperkte lees- en schrijfvaardigheden hebben, maar net zoals iedereen hebben zij zelfstandige mobiliteitsbehoeften voor o.a. verplaatsingen van en naar huis, een maatwerkbedrijf of een vrijetijdsbesteding. Lezen doen we allemaal zonder erbij stil te staan. Onze informatiemaatschappij gaat uit van dit basisprincipe. Alles is steeds meer digitaal, maar om te participeren en zelfstandig te functioneren, is lezen onmisbaar. De trein nemen, een busticket betalen, een plaats in de bioscoop boeken, info zoeken op het internet over een bepaalde locatie, ... al deze handelingen vereisen minimale geletterdheid. Deze doelgroep wordt dus snel geconfronteerd met uitdagingen bij het plannen en uitvoeren van verplaatsingen. Afhankelijk van hun type beperking kunnen deze noden bovendien sterk variëren. Bij autismespectrumstoornis (ASS) werden diverse aspecten omtrent tijdsbesef samengevat door Prof. Degrieck \autocite{Degrieck2014}. Een grote gemeenschappelijke deler bij deze doelgroep is de nood aan eenduidige structuur, visuele ondersteuning en afwezigheid van onnodige prikkels die afleiden van het tijdsbesef. De website van Participate\footnote{\url{https://nl.participate-autisme.be}} bundelt hieromtrent heel wat ervaringsgerichte informatie via hun FAQ en enkele blogs. Zeker in de context van ASS is een advertentieluwe omgeving met enkel essentiële informatie een absolute noodzaak \autocite{Roeyers2014}. De mogelijkheid om te kiezen voor een klassiek klokbeeld in plaats van het digitale beeld of een keuze voor bepaalde lettertypes kan voor deze mensen het verschil maken. Mensen met een laag IQ kunnen bovendien slechts bepaalde letter- en cijfercombinaties memoriseren \autocite{DeGraaf2001, Tytgat2014}. \textcite{Tilborg2018} bestudeerde hiertoe welke tekenen van geletterdheid specifiek voor kinderen met intellectuele beperkingen nuttig zijn. \textcite{Uyttersprot2021} baseerde hierop haar werk ‘Lees je honger’ waarin eten en drinken als basis aangewend worden voor functioneel en voorbereidend lezen voor kinderen in Type 2 buitengewoon onderwijs. Dit laatste werk geeft aan hoe vanuit een beperkte woordenschat, gebaseerd op voeding, specifieke letters kunnen aangeleerd worden via visuele herkenningspunten en zo toch enige leesvaardigheid kan worden bereikt. Door de POC af te stemmen op die beperkte leesvaardigheid kan de tool een oplossing vormen voor de specifieke doelgroep. Een bijkomende optie voor beide doelgroepen (beperkt intellectueel en adaptief vermogen) is het gebruik van standaard pictogrammen. Een algemeen erkend systeem zijn de zogenaamde Sclera picto’s \footnote{\url{https://www.sclera.be/nl/vzw/geschiedenis}}. Ze vormen een sterke visuele ondersteuning en zijn een benchmark in deze context. Sclera is een vzw opgericht in 2008 met als missie om bij te dragen tot het realiseren van communicatieve en verstandelijke toegankelijkheid. De pictogrammen ontwikkeld door Bart Serrien werden gedurende enkele jaren uitgebreid en aangeboden als hulpmiddel aan diverse organisaties en doelgroepen. Door via hun werking te sensibiliseren, te informeren en advies te formuleren, vonden deze pictogrammen hun weg naar het onderwijs en vele professionele ondersteunende organisaties. Hun website omvat nog steeds een ruim overzicht aan tools met andere pictogramsystemen, software, apps, online tools en advies of vorming. Veel mensen met een mentale beperking, en hun begeleiders, kennen deze pictogrammen. In de huidige state of the art is weinig tot geen informatie beschikbaar over het samen toepassen in mobiliteitstools van deze verschillende technieken om zo optimaal bij te dragen tot begrijpbaarheid en functionaliteit. De noodzaak van accurate locatie gebaseerde informatie, het kiezen van een prikkelarme weergave, de mogelijkheid van het gebruik van pictogrammen voor het beschrijven van de bestemming en de integratie van spraakherkenning voor gebruikers die niet kunnen lezen of schrijven, is nieuw in deze context.


\section{Augmented Reality}
\label{sec:augmented-reality}

Augmented Reality (AR) is een technologie die de gebruikerservaring verbetert door digitale informatie over objecten of plaatsen in de echte wereld te projecteren \autocite{Berryman2012}. Het verschilt van virtual reality (VR) doordat gebruikers nog steeds hun fysieke omgeving ervaren \autocite{Calo2015}. AR wordt op verschillende gebieden gebruikt, waaronder geneeskunde, marketing en onderwijs, en wordt gedefinieerd als een real-time weergave van de fysieke wereld, versterkt door virtueel computer gegenereerde informatie \autocite{Carmigniani2011}. Het is een veelzijdige technologie met potentieel positieve toepassingen, maar brengt ook beleidszorgen met zich mee die moeten worden aangepakt \autocite{Calo2015}. Er zijn verschillende soorten beleidsvormen. Het bouwen van dynamische systemen die flexibel zijn en kunnen bijgewerkt worden naarmate de technologie en cultuur veranderen. AR wordt anders ervaren door verschillende mensen dus het is belang- rijk diverse bevolkingsgroepen te raadplegen. De data bescherming is belangrijk naar wat kan en niet, zodat het systeem ook geen vooroordelen trekt op vlak van bepaalde verkeerde gevoede data. De algemene conclusie hieruit is dat AR-systemen ontwikkelt moeten worden zodanig dat deze gemakkelijk aanpasbaar naarmate de verandering in technologie en samenleving \autocite{Calo2015}. Zoals hiervoor vermeld kan de integratie van AR extra mogelijkheden toevoegen aan de gebruikservaring van mensen met een verstandelijke beperking die een beperkter inlevings- en inbeeldingsvermogen hebben. Het visualisatie aspect kan een hulpmiddel zijn voor mensen met adaptatieproblemen die moeilijk om kunnen met gewijzigde omstandigheden. Google Maps maakt reeds gebruik van een statische visualisatie. Hieraan zou AR een ‘live’ laag kunnen toevoegen door bijvoorbeeld het beeld aan te passen in functie van het seizoen. AR brengt specifieke beleidszorgen met zich mee die aangepakt moeten worden naarmate deze technologie zich verder ontwikkelt en integreert in diverse aspecten van het dagelijks leven. Hieronder zijn enkele belangrijke beleidsvormen die overwogen moeten worden:

\begin{itemize}
    \item \textbf{Privacybescherming}: Gezien AR-apparaten in staat zijn om continue gegevens over de gebruikers en hun omgeving te verzamelen, is het cruciaal dat er sterke privacyrichtlijnen en -beleid worden geformuleerd en geïmplementeerd. Dit omvat beleid dat reguleert hoe deze gegevens worden verzameld, gebruikt, en gedeeld. De bescherming van persoonlijke gegevens moet voorop staan om misbruik en ongeautoriseerde toegang te voorkomen.
    \item \textbf{Gelijkheid en non-discriminatie}: Beleid moet ervoor zorgen dat AR-technologieën toegankelijk zijn voor alle gebruikers, ongeacht socio-economische status, locatie of fysieke vermogens. Het beleid moet discriminatie voorkomen en inclusiviteit bevorderen.
    \item \textbf{Veiligheid en welzijn}: AR-toepassingen kunnen potentiële veiligheidsrisico's met zich meebrengen, zoals afleiding en desoriëntatie. Beleidsmakers moeten richtlijnen ontwikkelen die de veiligheid van gebruikers garanderen, vooral in situaties waarbij de fysieke en digitale wereld overlappen.
    \item \textbf{Flexibiliteit en aanpasbaarheid}: Beleid moet flexibel genoeg zijn om aanpassingen toe te staan naarmate de technologie en maatschappelijke normen evolueren. Dit zorgt ervoor dat wetgeving relevant en effectief blijft in het licht van snelle technologische vooruitgang.
    \item \textbf{Ethische overwegingen}: Het ontwikkelen en implementeren van AR moet geleid worden door ethische overwegingen die rekening houden met de impact op individuen en de samenleving als geheel. Dit omvat het overwegen van de effecten van AR op sociale interacties, mentale gezondheid, en persoonlijke autonomie.
\end{itemize}

Deze beleidsvormen en zorgen zijn van essentieel belang om te zorgen dat AR-technologieën op een verantwoorde en effectieve manier worden ingezet, waarbij de rechten en welzijn van individuen worden beschermd en bevorderd \autocite{Roesner2014}.



%TODO: "Er zijn verschillende soorten beleidsvormen.": welke beleidsvormen zijn er?
%TODO: ook de zinnen die erna komen zijn niet helemaal duidelijk, kan je deze verduidelijken? Ze spreken over zoveel dingen tegelijkertijd, en zo net te kort en bondig.

\section{Artificiële Intelligentie}
\label{sec:artificiele-intelligentie}
%TODO: referentie bij de definitie van AI
%TODO: bespreek meer in detail wat er in de bronnen staat (geldt eigenlijk voor alle secties)

Artificiële Intelligentie (AI) omvat het vermogen van machines of computersystemen om taken uit te voeren die traditioneel menselijke intelligentie vereisen, zoals logisch redeneren, leren, en problemen oplossen. AI-systemen gebruiken complexe algoritmen en technologieën voor machinaal leren om cognitieve vaardigheden na te bootsen, waardoor ze in staat zijn autonoom te functioneren in diverse omgevingen \autocite{Sabouret2020}. In de context van navigatie en routeplanning, stelt AI ons in staat om optimale routes te berekenen door real-time verkeersgegevens en geografische informatie te analyseren. Door gebruik te maken van geavanceerde datastructuren, kunnen AI-systemen locatie-informatie koppelen aan routepatronen, waardoor het mogelijk wordt knelpunten te identificeren en routes aan te passen aan veranderende omstandigheden zoals verkeersopstoppingen of wegwerkzaamheden \autocite{Soni2023a,Ruta2010}. Het belang van realtime gegevens is cruciaal in deze systemen, omdat deze de AI in staat stellen snel en efficiënt te reageren op dynamische veranderingen in de omgeving \autocite{Ciravegna2018}. Deze AI-gestuurde aanpak heeft het potentieel om de efficiëntie en nauwkeurigheid van navigatiesystemen aanzienlijk te verbeteren, waardoor gebruikers tijd kunnen besparen en hun reiservaring kunnen optimaliseren. Door voortdurend de nieuwste technologieën en methodieken te integreren, kunnen we verwachten dat AI een steeds belangrijkere rol zal spelen in de evolutie van intelligente transportsystemen.

%TODO: je stopt zo abrupt, kan je dit wat beter afronden?

\section{Hedendaagse vormen of factors van navigatie}
\label{sec:literatuuroverzicht}

%TODO: de titel van deze sectie is vreemd, die lijkt heel sterk op de BP van Tyrone maar houdt in het kader van navigatie eigenlijk geen steek. Je kan beter de titel van deze sectie aanpassen naar iets als "Hedendaagse navigatiemethodes" of iets dergelijks.
%TODO: niet meteen een subsectie beginnen, eerst een inleidende paragraaf
% je kan hier bespreken wat je gaat doen in deze sectie, waar je naar gaat kijken, of je bepaalde tools al uitsluit (en waarom), enz.
%TODO: elk van de tools bespreek je heel kort en bonding, je springt in een alinea van de hak op de tak soms. Je kan nooit te veel schrijven, schrijf gerust zoveel je wil en behandel alles in detail. Dat maakt de tekst ook vlotter leesbaar.

In deze sectie onderzoeken we verschillende hedendaagse navigatiemethodes die gebruikt worden om locaties te vinden, routes te plannen, en navigatie-instructies te verstrekken. De focus ligt op de bruikbaarheid van deze technologieën voor mensen met een beperking, en bespreken de toegankelijkheid en aanpasbaarheid van elk systeem.

\subsection{Google Maps}
\label{sec:google-maps}

 %TODO: beschrijf eerst wat Google Maps is alvorens je de voor- en nadelen opsomt, idem bij de andere tools

Google Maps is een gebruikersvriendelijke ervaring voor gewone gebruikers om locaties te zoeken, routes te plannen en navigatie-instructies te ontvangen. Een standaardprocedure voor een normaal persoon zou zijn:

\begin{enumerate}
    \item De gebruiker geeft een de naam van de bestemming in.
    \item Hierna kan de gebruiker kiezen voor route waarbij Google Maps automatisch de best mogelijke routes weergeeft.
    \item Als laatste geeft de applicatie onderweg stapsgewijze instructies met auditieve geluiden of visuele kaartbegeleiding.
\end{enumerate}

Voor personen met een mentale beperking kan in deze context de cognitieve overbelasting een mogelijke blokkade zijn. De complexiteit van de interface en de hoeveelheid informatie die ze terug krijgen kan mogelijke hinder vormen in het begrijpen en verwerken van deze informatie. Navigatieverwarring kan hierbij een ander probleem vormen. Bepaalde instructies zijn te complex of onduidelijk opgebouwd.

%TODO: wat "hele goede visuele beelden opleveren" is een vreemde verwoording
is de privacy. Het gebruik maken van real-time gegevens komt ten koste van heel wat locatie data. Een internetverbinding is vereist voor het gebruiken van deze applicatie. Als laatste voegde Google onlangs hun AR technologie: GoogleARVR² toe aan de applicatie Google Maps. Je kan deze gebruiken voor bekende plekken te bezoeken of gaan verkennen van de omgeving in AR. Dit vormt duidelijk een voordeel voor mensen met vooral adaptieve problematiek die hiervan gebruik kunnen maken om de omgeving van de bestemming op voorhand te verkennen.

\subsection{Mapbox}
\label{sec:mapbox}

Mapbox biedt een platform voor het maken en integreren van aangepaste kaarten en locatiegebaseerde services. De kaartweergave van Mapbox biedt verschillende kaartstijlen en aanpasbare kaartlagen, waardoor gebruikers een breed scala aan visuele informatie kunnen verkennen. Dit platform is ontworpen voor applicaties van derden, zo kunnen de navigatiediensten geïntegreerd worden in hun applicaties voor routebegeleiding en route-algoritmen. Mapbox heeft als groot voordeel de aanpasbaarheid per gebruiksgeval, zo kan het gemakkelijk aangepast worden naar specifieke behoeften en gebruikerservaringen. Het platform biedt AR/VR toepassingen aan, dit versterkt de leer mogelijkheden voor mensen met mentale beperking. Het grootste nadeel in deze applicatie is de afhankelijkheid van een ontwikkelaar. Ook de complexiteit die uit de applicatie komt kan mogelijke vaardigheden vereisen.

\subsection{Sygic}
\label{sec:sygic}

Sygic is een navigatie-app die gebruikers helpt bij het plannen van routes, navigeren op de weg en het vermijden van verkeersopstoppingen. De standaardprocedure voor het platform is je geeft je bestemming in en het programma berekent de beste route op basis van actuele verkeersinformatie. Tijdens het verplaatsen ontvangt de gebruikers stapsgewijze instructies aan de hand van spraaknavigatie, meldingen voor afslagen, rotondes en andere manoeuvres. Eens je jouw route hebt ingeladen kan je de verplaatsing offline verder zetten zonder actuele internetverbinding. Sommige geavanceerde functie van Sygic, zoals bepaalde voorkeuren instellen of het vinden van een specifiek punt %todo hier ontbreekt een stuk zin. 
Het grootste probleem met Sygic is dat de software gebouwd is voor auto navigatie.

\subsection{De Lijn}
\label{sec:de-lijn}

De Lijn staat in voor alles rond het bus- en tramnetwerk in Vlaanderen. Hun applicatie werkt als volgt: je geeft je vertrek- en aankomstpunt in. De applicatie geeft dan een lijst terug met mogelijke bussen of trams om efficiënt op jouw bestemming te geraken. Bij het ingeven van je vertrek- en aankomstpunt geeft het ingebouwde systeem ook meteen de dichtstbijzijnde halte weer en hoe ver het wandelen is van deze halte tot je bestemming. Het systeem biedt ook real-time gegevens aan over de bussen en trams. De complexiteit van de applicatie kan een cruciale rol spelen in het verplaatsen van zichzelf en de internetverbinding die vereist is.

\subsection{NMBS}
\label{sec:nmbs}

Nationale Maatschappij der Belgische Spoorwegen (NMBS) is de nationale treindienst van België. De grootste blokkade hier begint al bij het kennen van de vertrek en aankomsthalte. Deze applicatie biedt wel enkele voordelen zoals real-time informatie van de treintijden en het ticket systeem is in staat tickets online te bewaren. Specifieke problemen die zich hier kunnen voordoen is de complexiteit van de interface van de applicatie en een internetverbinding is vereist voor het gebruik van dit platform. In het algemeen loopt deze applicatie zeer analoog met die van De Lijn.

\section{Requirements van de gezochte navigatietool}
\label{sec:requirements van de gezochte navigatietool}

\subsection{Wat is MoSCoW?}
De MoSCoW-methode is een prioriteringsframework gebruikt in projectbeheer om projectvereisten te categoriseren in vier categorieën: Must have, Should have, Could have, en Won't have. Deze methode bevordert een efficiënte middelentoewijzing en zorgt ervoor dat essentiële taken prioriteit krijgen. Dit is bijzonder nuttig binnen Agile projectbeheer voor het beheer van de projectomvang en het verbeteren van de productiviteit door duidelijk niet-essentiële elementen die niet worden aangepakt te definiëren \autocite{Brush2023}.

\subsection{Functionele requirements}
\label{sec:functionele requirements}

\subsection*{Must have}
\begin{itemize}[label={--}]
    \item \textbf{Algemene bruikbaarheid}: Het systeem moet toegankelijk zijn voor alle gebruikers, ongeacht hun specifieke beperkingen.
    \item \textbf{Platform}: Moet ontwikkeld worden in React Native om compatibiliteit met diverse mobiele platformen te garanderen.
    \item \textbf{Intuïtiviteit}: Het ontwerp moet duidelijk en eenvoudig zijn, met minimale complexiteit.
    \item \textbf{Platformonafhankelijkheid}: Het systeem moet functioneren onafhankelijk van het mobiele besturingssysteem.
    \item \textbf{Visuele aanwijzingen}: Duidelijke visuele aanwijzingen zoals symbolen, pijlen of kleurcoderingen moeten de gebruiker door de route leiden.
    \item \textbf{Auditieve instructies}: Moet gesproken aanwijzingen of geluidssignalen bieden voor gebruikers die visuele informatie moeilijk kunnen verwerken.
    \item \textbf{Real-time updates}: Continue updates over de locatie van de gebruiker en aanpassingen aan veranderende omstandigheden zijn essentieel.
    \item \textbf{Toegankelijkheidsfuncties}: Ondersteuning voor schermlezers, aanpasbare lettertypen en kleuren, en vergrotingsmogelijkheden.
    \item \textbf{Connectiviteit}: Het systeem moet functioneel blijven bij kleine onderbrekingen in de internetverbinding.
\end{itemize}

\subsection*{Should have}
\begin{itemize}[label={--}]
    \item \textbf{Aanpasbaarheid}: Het systeem moet aanpasbaar zijn aan de individuele behoeften van de gebruiker, zoals voorkeursroutes en het vermijden van bepaalde obstakels.
    \item \textbf{Offline functionaliteit}: Het systeem moet bepaalde functies offline kunnen uitvoeren.
\end{itemize}

\subsection*{Could have}
\begin{itemize}[label={--}]
    \item \textbf{Veiligheidsfuncties}: Ingebouwde waarschuwingen voor gevaarlijke gebieden en suggesties voor veiligere routes.
\end{itemize}

\subsection*{Won't have}
\begin{itemize}[label={--}]
    \item \textbf{Visual noise}: Het systeem mag geen visuele of auditieve verstoringen veroorzaken die de gebruiker kunnen afleiden.
    \item \textbf{Speciale hardware}: Het systeem vereist geen speciale hardware voor augmented reality (AR) functies.
\end{itemize}

\subsection{Niet-functionele requirements}
\label{sec:niet-functionele-requirements}

Zoals eerder besproken, zullen hier de niet-functionele requirements worden behandeld. Deze zijn de kwalitatieve aspecten die de navigatiemethodes moeten bevatten om de ervaring voor de gebruiker zo aangenaam mogelijk te maken.

\subsubsection{Must have requirements}
\begin{itemize}
    \item Het navigatieproces moet snel zijn, om tijdverlies en onzekerheid te minimaliseren.
    \item De informatie die de gebruiker moet onthouden moet beperkt zijn, om het risico op fouten te verlagen.
    \item Het aantal stappen in het navigatieproces moet tot een minimum beperkt blijven.
\end{itemize}

\subsubsection{Should have requirements}
\begin{itemize}
    \item De navigatiemethodes moeten gebruiksvriendelijk zijn voor mensen met dyslexie. Er moet rekening gehouden worden met de lees- en onthoudmoeilijkheden die zij kunnen ervaren.
\end{itemize}

\subsubsection{Could have requirements}
\begin{itemize}
    \item Het zou voordelig zijn als de navigatiemethodes kostenefficiënt zijn, zodat onnodige kosten vermeden worden, wat de dienst toegankelijker maakt voor de eindgebruiker en de aanbieder.
\end{itemize}

\section{Het bepalen van de geschikte navigatiemethode}
\label{sec:bepalen geschikte navigatiemethode}

In deze sectie evalueren we verschillende navigatiemethodes om de meest geschikte methode te identificeren voor mensen met een mentale beperking. Het doel is om een methode te vinden die de zelfstandigheid van de gebruiker maximaliseert en tegelijkertijd de cognitieve belasting minimaliseert.

\subsection{Criteria voor Evaluatie}
\begin{itemize}
    \item \textbf{Gebruiksgemak:} Hoe intuïtief en eenvoudig is de interface voor de gebruiker?
    \item \textbf{Toegankelijkheid:} Biedt de methode ondersteuning voor toegankelijkheidsfuncties zoals spraakherkenning, aanpasbare tekstgrootte, en visuele of auditieve hulpmiddelen?
    \item \textbf{Reactievermogen:} Hoe goed kan de methode omgaan met veranderingen in de omgeving of bij onvoorziene omstandigheden?
    \item \textbf{Privacy:} Hoe gaat de methode om met gebruikersgegevens en privacybescherming?
\end{itemize}

\subsection{Evaluatie van Methodes}
We beschouwen verschillende bestaande technologieën, zoals Google Maps, Mapbox en gespecialiseerde apps zoals Sygic, en beoordelen deze op bovenstaande criteria. De methodes worden ook getest in een reële omgeving om hun effectiviteit in praktijksituaties te bepalen. We zullen eerst een nulmeting uitvoeren...

\subsection{Aanbevolen Methode}
Na grondige evaluatie en praktijktests wordt de meest geschikte navigatiemethode aanbevolen. Deze aanbeveling is gebaseerd op de mate waarin de methode voldoet aan de criteria en de feedback van testgebruikers.

\subsection{Implementatievoorstel}
We stellen een gedetailleerd plan voor voor de implementatie van de aanbevolen navigatiemethode, inclusief stappen voor verdere ontwikkeling en integratie met bestaande systemen om de toegankelijkheid voor de doelgroep te vergroten.

\section{Technologieën}
\label{sec:technologieën}

%TODO: ook hier moet je een inleidende paragraaf voorzien

In dit gedeelte wordt een overzicht gegeven van verschillende technologieën die van belang zijn voor het ontwikkelen van de applicatie. Technologieën evolueren voortdurend en spelen een cruciale rol in het bepalen van de functionaliteit, prestaties en gebruiksvriendelijkheid van softwaretoepassingen. Het selecteren van de juiste technologieën is essentieel voor het bereiken van de gewenste resultaten en het voldoen aan de behoeften van gebruikers voor navigatie. Daarom worden hier enkele van de belangrijkste technologieën besproken die worden gebruikt in het ontwikkelingsproces van de applicatie of POC, waaronder JavaScript en React Native. Deze technologieën bieden krachtige tools en frameworks die ontwikkelaars in staat stellen om dynamische en responsieve gebruikersinterfaces te bouwen, zowel voor webapplicaties als voor mobiele apps.

\subsection{Javascript}
\label{sec:javascript}

%TODO: de "U" en "uw" in deze tekst zijn heel vreemd, de schrijfstijl verschilt van de rest van de tekst.

JavaScript is een scripttaal waarmee je statische webapplicaties kan verbeteren met dynamische, gepersonaliseerde en interactieve inhoud. Dit verbetert de ervaring van bezoekers op uw site en maakt het waarschijnlijker dat ze opnieuw langskomen. U hebt vast de flikkerende uitklapmenu's, bewegende tekst en veranderende inhoud die nu wijdverspreid zijn op websites. Ze worden mogelijk gemaakt door JavaScript. JavaScript wordt ondersteund door alle grote browsers en is de taal bij uitstek op het web. Het kan zelfs worden gebruikt buiten webapplicaties, bijvoorbeeld om administratieve taken te automatiseren \autocite{Wilton2004}.

\subsection{React Native}
\label{sec:react native}

React is een bibliotheek waarmee ontwikkelaars gebruikersinterfaces (UI's) kunnen bouwen als een boom van kleine stukjes die componenten worden genoemd. Een component is een mix van HTML en JavaScript die alle logica bevat die nodig is om een klein deel van een grotere UI weer te geven. Elk van deze componenten kan worden opgebouwd tot opeenvolgende complexe onderdelen van een app \autocite{Baer2018}. React Native is een populair open-source framework, dat is ontwikkeld door Facebook. Ontwikkelaars kunnen hiermee mobiele applicaties bouwen aan de hand van JavaScript. React Native applicaties gebruiken de kracht van React, een JavaScript bibliotheek, voor het bouwen van gebruikersinterfaces met mobiele componenten. Door deze componenten kunnen ontwikkelaars efficiënt applicaties maken die werken voor zowel iOS als Android apparaten \autocite{Vinnik2021}. De keuze voor React Native als framework voor het ontwikkelen van mobiele applicaties is gebaseerd op verschillende factoren. Allereerst biedt React Native een efficiënte manier om cross-platform mobiele apps te bouwen met behulp van een enkele codebase, wat resulteert in een kosten- en tijdsbesparing. Bovendien maakt het gebruik van JavaScript het gemakkelijker voor ontwikkelaars om snel te leren en beginnen met het bouwen van mobiele applicaties. Tot slot heeft React Native een actieve community en een groot aantal beschikbare bibliotheken en plugins, waardoor ontwikkelaars toegang hebben tot een breed scala aan functionaliteiten en mogelijkheden voor het uitbreiden van hun applicaties.

%TODO: Beschrijf hier ook waarom je React Native hebt gekozen, en niet een andere technologie

%TODO: In het algemeen moet je voor hoofdstuk 2 zeker nog eens het deel A.2 overlopen en checken welke referentie hierboven nog niet gebruikt zijn. Die moeten ook nog een plek vinden in een van de paragrafen hierboven. Ik zou ook eens vragen aan Aelbrecht of het de bedoeling is om een soort van conclusie te schrijven aan het einde van dit stuk. 
