\chapter{\IfLanguageName{dutch}{Stand van zaken}{State of the art}}%
\label{ch:stand-van-zaken}

% Tip: Begin elk hoofdstuk met een paragraaf inleiding die beschrijft hoe
% dit hoofdstuk past binnen het geheel van de bachelorproef. Geef in het
% bijzonder aan wat de link is met het vorige en volgende hoofdstuk.

% Pas na deze inleidende paragraaf komt de eerste sectiehoofding.

%%\lipsum[7-20]

Dit hoofdstuk bevat je literatuurstudie. De inhoud gaat verder op de inleiding, maar zal het onderwerp van de bachelorproef *diepgaand* uitspitten. De bedoeling is dat de lezer na lezing van dit hoofdstuk helemaal op de hoogte is van de huidige stand van zaken (state-of-the-art) in het onderzoeksdomein. Iemand die niet vertrouwd is met het onderwerp, weet nu voldoende om de rest van het verhaal te kunnen volgen, zonder dat die er nog andere informatie moet over opzoeken \autocite{Pollefliet2011}.

Je verwijst bij elke bewering die je doet, vakterm die je introduceert, enz.\ naar je bronnen. In \LaTeX{} kan dat met het commando \texttt{$\backslash${textcite\{\}}} of \texttt{$\backslash${autocite\{\}}}. Als argument van het commando geef je de ``sleutel'' van een ``record'' in een bibliografische databank in het Bib\LaTeX{}-formaat (een tekstbestand). Als je expliciet naar de auteur verwijst in de zin (narratieve referentie), gebruik je \texttt{$\backslash${}textcite\{\}}. Soms is de auteursnaam niet expliciet een onderdeel van de zin, dan gebruik je \texttt{$\backslash${}autocite\{\}} (referentie tussen haakjes). Dit gebruik je bv.~bij een citaat, of om in het bijschrift van een overgenomen afbeelding, broncode, tabel, enz. te verwijzen naar de bron. In de volgende paragraaf een voorbeeld van elk.

\textcite{Knuth1998} schreef een van de standaardwerken over sorteer- en zoekalgoritmen. Experten zijn het erover eens dat cloud computing een interessante opportuniteit vormen, zowel voor gebruikers als voor dienstverleners op vlak van informatietechnologie~\autocite{Creeger2009}.

Let er ook op: het \texttt{cite}-commando voor de punt, dus binnen de zin. Je verwijst meteen naar een bron in de eerste zin die erop gebaseerd is, dus niet pas op het einde van een paragraaf.

De literatuurstudie zal zich richten op het verkennen van de mobiliteitsbehoeften van de doelgroep en het bespreken van bestaande tools en technologieën die hen kunnen ondersteunen bij het navigeren en reizen.
  Mensen met een beperking worden vaak geconfronteerd met uitdagingen bij het plannen en uitvoeren van verplaatsingen. Afhankelijk van hun type beperking kunnen deze noden bovendien sterk variëren. Bij autismespectrumstoornis (ASS) werden diverse aspecten omtrent tijdsbesef samengevat door Prof. Degrieck \autocite{Degrieck2014}.
  Een grote gemeenschappelijke deler bij deze doelgroep is de nood aan eenduidige structuur, vaak visueel ondersteund en de afwezigheid van onnodige prikkels. De website van Participate\footnote{\url{https://nl.participate-autisme.be}} bundelt hieromtrent heel wat ervaringsgerichte informatie via hun FAQ en enkele blogs. Zeker in de context van ASS is een advertentie-luwe omgeving met enkel essentiële informatie een absolute noodzaak \autocite{Roeyers2014}. 
  De mogelijkheid om te kiezen voor een klassiek klokbeeld in plaats van het digitale beeld of een keuze voor bepaalde lettertypes kan voor mensen met een laag IQ die slechts bepaalde letter- en cijfercombinaties kunnen memoriseren het verschil maken\\ \autocite{Uyttersprot2021,Tytgat2014,DeGraaf2001}. \\Welke technologieën dragen het beste bij tot begrijpbaarheid en kunnen zo hun mobiliteit en zelfstandigheid verbeteren?
  Een aantal belangrijke aandachtspunten en best practices voor mensen met een beperking zullen vanuit deze invalshoeken worden onderzocht, zoals de noodzaak van locatiegebaseerde informatie, de mogelijkheid van het gebruik van pictogrammen voor het beschrijven van locaties, en de integratie van spraakherkenning voor gebruikers die niet kunnen lezen of schrijven.Daarnaast zullen verschillende soorten mobiliteitsapps worden besproken, waaronder applicaties van openbare vervoersmaatschappijen zoals NMBS\footnote{\url{https://www.belgiantrain.be/nl}} en De Lijn\footnote{\url{https://www.delijn.be/nl/}}, standaard routeplanners en andere specifieke mobiliteitsapps zoals Dott\footnote{\url{https://ridedott.com/nl/}} en Villo\footnote{\url{https://www.villo.be/nl/home}} voor fietsen en steps. Het doel is om te analyseren hoe deze apps momenteel functioneren en welke verbeteringen nodig zijn om ze toegankelijker te maken voor mensen met een beperking.
  Deze applicaties en websites zijn gebaseerd op het principe dat de gebruiker zelfstandig een adres en tijdstip zelf kan ingeven en maken niet altijd gebruik van locatiebepaling, wat voor de doelgroep een duidelijke hinderpaal is.
  Daarnaast zullen diverse Augmented Reality (AR) methodes binnen de huidige state of the art vergeleken worden voor het verbeteren van de gebruikerservaring.
  Mapbox AR\footnote{\url{https://www.mapbox.com/augmented-reality}} maakt gebruik van points of interest terwijl Google Maps AR \footnote{\url{https://arvr.google.com}} inzet op multidimensionele visualisatie om de gebruiker comfortabel aan te sturen tijdens het navigeren.
  Met behulp van de camera van het mobiele apparaat worden real-time beelden van de omgeving vastgelegd, waarbij digitale routeaanwijzingen over de werkelijke beelden worden geprojecteerd.
  Dit zorgt voor een intuïtieve en praktische navigatie-ervaring, vooral in stedelijke gebieden waar traditionele kaarten mogelijk minder effectief zijn.
  Mapbox AR onderscheidt zich door zijn typische karakteristiek van geavanceerde aanpasbaarheid en integratiemogelijkheden.
  Deze tool biedt ontwikkelaars een krachtig platform waarmee ze op maat gemaakte augmented reality-toepassingen kunnen creëren, variërend van navigatie tot locatiegebaseerde informatie.
  Hierdoor hebben ontwikkelaars de flexibiliteit om kaartgegevens aan te passen, aangepaste overlays toe te voegen en de gebruikerservaring te optimaliseren voor specifieke doeleinden.
  Beide tools vertegenwoordigen technologieën binnen het domein van augmented reality en kaartnavigatie. Ze tonen de evolutie aan van traditionele kaartapplicaties naar meer dynamische, op AR gebaseerde oplossingen.
  Ook Koombea\footnote{\url{https://www.koombea.com}} verhoogt de gebruikerservaring en met AR City\footnote{\url{https://arcitygame.nl}} komt de locatie effectief tot leven dankzij bijkomende informatie.
  Sygic\footnote{\url{https://www.sygic.com}} werd specifiek ontwikkeld voor auto GPS-systemen om de bestuurderservaring te optimaliseren, maar kan een toegevoegde waarde hebben wanneer de gebruiker tijdens een busrit het traject mee wil opvolgen.
  De implementatie van locatiegevoelige informatie in combinatie met de AR visualisatie neemt diverse barrières weg voor de doelgroep.
  Zij hoeven immers niet het adres te kennen van hun startpunt. Het eindpunt kunnen ze bv. omschrijven via een pictogram zoals deze gebruikt worden in Sclera\footnote{\url{https://www.sclera.be/nl/picto/overview}}.
  Deze pictogrammen worden standaard aangeleerd bij onze doelgroep ter bevordering van hun communicatie en visualisatie van hun dagplanning.
  Naast het openbaar vervoer kunnen andere systemen zoals Dott en Villo worden overwogen omdat dergelijke fietsen en steps verspreid staan in diverse steden.
  Ze bieden een bijkomende optie voor het verplaatsen van punt A naar punt B op een efficiënte manier.
  Op hardwaregebied ontwikkelde Google in 2014 een smart device genaamd Google Glasses.
  Deze brillen hadden als doel om AR een extra boost te geven aan de hand van een concreet concept.
  Het product kreeg een tweede versie in 2017, als ondernemingseditie, maar had geen succes en eerder dit jaar in maart 2023 kondigde Google aan dat ze het project stopzetten \autocite{Gvora2023}.
  Tot slot geldt dat de doelgroep die niet in staat is te lezen of schrijven, wel gebruik kan maken van text-to-speech functionaliteit om de gewenste locatie te omschrijven.
  De tool moet met andere woorden ook deze stemherkenning als ondersteuning bieden.
  Dankzij artificiële intelligentie (AI) wordt dan de relatief beste route berekent en tevens de beste manier om daar eenvoudig te geraken \autocite{Soni2023a}.
  Het gebruik van datastructuren zal toelaten deze locatie informatie te linken aan het routepatroon \autocite{Ruta2010}.
  Bovendien kunnen deze patronen potentiële knelpunten identificeren, hulp bieden bij het optimaliseren van routes als er verkeerswerken of files zijn.
  De noodzaak van dit systeem zijn realtime gegevens waarbij samenwerking met belanghebbende belangrijk is \autocite{Ciravegna2018}.
  Het opzetten van een meta model kan begeleiders van mensen met een matige mentale beperking helpen met het navigeren naar een juiste plaats.
  Door gegevensverzameling kan er aan de hand van algoritmes voorspellingen gemaakt worden wat voor deze persoon de favoriete manier van verplaatsen is of favoriete route is om te volgen naar het werk \autocite{Stepanov2003}.
  Als laatste kan er gebruik gemaakt worden van Internet of Things (IoT) tools voor het slim communiceren tussen verschillende apparatuur wat kan leiden tot nog betere prestaties in verplaatsing en routeberekening \autocite{Fatnassi2015}.
