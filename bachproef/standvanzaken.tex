\chapter{\IfLanguageName{dutch}{Stand van zaken}{State of the art}}%
\label{ch:stand-van-zaken}
% Tip: Begin elk hoofdstuk met een paragraaf inleiding die beschrijft hoe
% dit hoofdstuk past binnen het geheel van de bachelorproef. Geef in het
% bijzonder aan wat de link is met het vorige en volgende hoofdstuk.

% Pas na deze inleidende paragraaf komt de eerste sectiehoofding.

%%Dit hoofdstuk bevat je literatuurstudie. De inhoud gaat verder op de inleiding, maar zal het onderwerp van de bachelorproef *diepgaand* uitspitten. De bedoeling is dat de lezer na lezing van dit hoofdstuk helemaal op de hoogte is van de huidige stand van zaken (state-of-the-art) in het onderzoeksdomein. Iemand die niet vertrouwd is met het onderwerp, weet nu voldoende om de rest van het verhaal te kunnen volgen, zonder dat die er nog %%andere informatie moet over opzoeken \autocite{Pollefliet2011}.

%%Je verwijst bij elke bewering die je doet, vakterm die je introduceert, enz.\ naar je bronnen. In \LaTeX{} kan dat met het commando \texttt{$\backslash${textcite\{\}}} of \texttt{$\backslash${autocite\{\}}}. Als argument van het commando geef je de ``sleutel'' van een ``record'' in een bibliografische databank in het Bib\LaTeX{}-formaat (een tekstbestand). Als je expliciet naar de auteur verwijst in de zin (narratieve referentie), gebruik je \texttt{$\backslash${}textcite\{\}}. Soms is de auteursnaam niet expliciet een onderdeel van de zin, dan gebruik je \texttt{$\backslash${}autocite\{\}} (referentie tussen haakjes). Dit gebruik je bv.~bij een citaat, of om in het bijschrift van een overgenomen afbeelding, broncode, tabel, enz. te %%verwijzen naar de bron. In de volgende paragraaf een voorbeeld van elk.

%%\textcite{Knuth1998} schreef een van de standaardwerken over sorteer- en zoekalgoritmen. Experten zijn het %%erover eens dat cloud computing een interessante opportuniteit vormen, zowel voor gebruikers als voor %%dienstverleners op vlak van informatietechnologie~\autocite{Creeger2009}.

%%Let er ook op: het \texttt{cite}-commando voor de punt, dus binnen de zin. Je verwijst meteen naar een bron in %%de eerste zin die erop gebaseerd is, dus niet pas op het einde van een paragraaf.

%%\lipsum[7-20]
\section{Navigatie}
\label{sec:navigatie}
Navigatie is de kunst van het plannen en volgen van een route om zich daarmee van de huidige positie naar de bestemming te verplaatsen. Het woord 'navigatie' is afgeleid uit de Latijnse woorden navis, dat schip betekent, en agere, dat in deze context bewegen of sturen betekent. Bij navigatie gaat het erom dat je je eigen plaats bepaald en de plaats bepaald waar je heen gaat en daarna de weg er naar toe. \textcite{Katz2010} introduceerde het NAVIG-systeem, dat GNSS (Global Navigation Satellite System) en snelle visuele herkenning combineert om visueel gehandicapte gebruikers te begeleiden in zowel bekende als onbekende omgevingen. Deze systemen benadrukken het belang van technologie voor het verbeteren van navigatie voor mensen met een beperking. Om hier iets aan te doen, ontwikkelde \textcite{Lakde2015} een navigatiehulpsysteem voor visueel gehandicapten, dat gebruik maakt van een combinatie van sensortechnologie en stembegeleiding. Dit systeem maakt gebruikers bewust van hun pad en eventuele obstakels. Navigatie, zoals gedefinieerd door \textcite{Gachet2010}, is de handeling van het bewegen door een onbekende omgeving, en kan met name een uitdaging zijn voor visueel gehandicapten.
\section{Mentale Beperking}
\label{sec:mentale beperking}
\section{Augmented Reality}
\label{sec:augmented reality}
Augmented Reality (AR) is een technologie die de gebruikerservaring verbetert door digitale informatie over objecten of plaatsen in de echte wereld te projecteren (Berryman, 2012). Het verschilt van virtual reality (VR) doordat gebruikers nog steeds hun fysieke omgeving ervaren \autocite{Calo2015}. AR wordt op verschillende gebieden gebruikt, waaronder geneeskunde, marketing en onderwijs, en wordt gedefinieerd als een realtime weergave van de fysieke wereld, versterkt door virtueel computergegenereerde informatie \autocite{Carmigniani2011}. Het is een veelzijdige technologie met potentiële positieve toepassingen, maar brengt ook beleidszorgen met zich mee die moeten worden aangepakt \autocite{Calo2015}. Er zijn verschillende soorten beleidsvormen. Het bouwen van dynamische systemen die flexibel zijn en kunnen bijgewerkt worden naarmate de technologie en cultuur veranderen. AR wordt anders ervaren voor verschillende mensen dus het is belangrijk diverse bevolkingsgroepen te raadplegen. De data bescherming is belangrijk naar wat kan en niet zodat het systeem ook geen vooroordelen trekt op vlak van bepaalde verkeerde gevoede data. De algemene conclusie hieruit is dat AR-systemen ontwikkelt moeten worden zodanig dat deze gemakkelijk aanpasbaar naarmate de verandering in technologie en samenleving \autocite{Calo2015}.
\section{Artificiële Intelligentie}
\label{sec:artificiële intelligentie}
Artificiële Intelligentie (AI) is het vermogen van een machine of computersysteem om taken te simuleren en uit te voeren die menselijke intelligentie vereisen, zoals logisch redeneren, leren en problemen oplossen. AI is gebaseerd op het gebruik van algoritmen en technologieën voor machinaal leren om machines in staat te stellen bepaalde cognitieve vaardigheden na te bootsen. Naarmate AI zich verder ontwikkelt, kan het naar verwachting de efficiëntie van veel processen verbeteren en mensen helpen om complexe taken sneller en nauwkeuriger uit te voeren \autocite{Sabouret2020}. AI kan het mogelijk maken in ons gebied van onderzoek de relatief beste route te bereken en tevens de beste manier om eenvoudig op de bestemming te geraken \autocite{Soni2023a}. Het gebruik van datastructuren zal toelaten deze locatie informatie te linken aan het routepatroon \autocite{Ruta2010}. Deze patronen kunnen potentiële knelpunten identificeren, hulp bieden bij het optimaliseren van routes als er verkeerswerken of files zijn. De noodzaak van dit systeem zijn realtime gegevens \autocite{Ciravegna2018}.
\section{Hedendaagsevormen of factors van navigatie}
\label{sec:literatuuroverzicht}
\subsection{Google maps}
\label{sec:google maps}
\subsection{Mapbox}
\label{sec:mapbox}
\subsection{Sygic}
\label{sec:sygic}
\subsection{GoogleARVR}
\label{sec:googlearvr}
\subsection{De Lijn}
\label{sec:delijn}
\subsection{NMBS}
\label{sec:nmbs}
\subsection{Koombea}
\label{sec:koombea}
\section{Requirements}
\label{sec:requirements}
Bekijken welke tools voldoen aan de beste requirements 1-2 tools
\section{Het bepalen van de geschikte navigatiemethode}
\label{sec:het bepalen van de geschikte navigatiemethode}
\section{Technologieën}
\label{sec:technologieën}

Cross platform applicaties

basis intro javascript, typescript, react narratieve


POC opstarten