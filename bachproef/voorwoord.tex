%%=============================================================================
%% Voorwoord
%%=============================================================================

\chapter*{\IfLanguageName{dutch}{Woord vooraf}{Preface}}%
\label{ch:voorwoord}

%% TODO:
%% Het voorwoord is het enige deel van de bachelorproef waar je vanuit je
%% eigen standpunt (``ik-vorm'') mag schrijven. Je kan hier bv. motiveren
%% waarom jij het onderwerp wil bespreken.
%% Vergeet ook niet te bedanken wie je geholpen/gesteund/... heeft

%% \lipsum[1-2]

%- literatuurstudie
%- requirementsanalyse
%- interview
%- probleemstelling
%- doelstelling
%- methodologie in bap voorstel plakken -> feedback wachten
%- inleiding onderzoeksvraag en deelvragen alleen rest op het einde
%- voorwoord -> voor op het einde.
Met trots en dankbaarheid presenteer ik u het voorwoord voor mijn bachelorproef met als onderwerp een vergelijkende studie van Mapbox en Google Maps als navigatietools voor personen met een mentale beperking. 

Mijn jongere 14-jarige broer, die noch kan lezen of schrijven, hoort thuis in deze doelgroep en vormde de rechtstreekse aanleiding voor de keuze van dit onderwerp. Het is voor hem niet eenvoudig om zelfstandig te navigeren van punt A naar punt B zonder deze vanzelfsprekende vaardigheden. Ik wilde hiermee dit onderwerp meer bespreekbaar maken en hoop dat ik anderen er mee kan inspireren om naar inclusieve oplossingen te zoeken. Het grootste struikelblok bleek het ontwikkelen van een eigen applicatie die voldeed aan de requirements van de doelgroep. Het onderzoek heeft echter zeker bijgedragen aan het verwerven van duidelijke inzichten in deze doelgroep en de criteria voor de meest geschikte methode om te kunnen navigeren. 

Het schrijven van deze scriptie was niet zonder uitdagingen en daarom, wil ik mijn promotor bedanken voor de nodige feedback. Ik wil ook van harte mijn co-promotor bedanken die mijn vele versies nalas en waardevolle feedback gaf. Ons tweewekelijks overleg verzekerde de nodige bijsturing en hij zorgde voor de middelen die nodig waren voor dit onderzoek. Dankzij zijn kennis en duiding werd mijn bachelorproef in goede banen geleid. Ook wil ik mijn mama bedanken voor het nalezen van de vele versies en het opleveren van waardevolle feedback als ervaringsdeskundige ouder van een kind met een beperking.

Door dit proces heb ik kennis en vaardigheden kunnen verwerven, die ik in mijn toekomstige carrière zal gebruiken. Het afronden van deze scriptie markeert een belangrijke mijlpaal in mijn academische jaren en vormt een stevige basis voor mijn verdere professionele ontwikkeling.

Robin De Waegeneer
