%%=============================================================================
%% Voorwoord
%%=============================================================================

\chapter*{\IfLanguageName{dutch}{Woord vooraf}{Preface}}%
\label{ch:voorwoord}

%% TODO:
%% Het voorwoord is het enige deel van de bachelorproef waar je vanuit je
%% eigen standpunt (``ik-vorm'') mag schrijven. Je kan hier bv. motiveren
%% waarom jij het onderwerp wil bespreken.
%% Vergeet ook niet te bedanken wie je geholpen/gesteund/... heeft

%% \lipsum[1-2]

%- literatuurstudie
%- requirementsanalyse
%- interview
%- probleemstelling
%- doelstelling
%- methodologie in bap voorstel plakken -> feedback wachten
%- inleiding onderzoeksvraag en deelvragen alleen rest op het einde
%- voorwoord -> voor op het einde.

Dit onderzoek gaat over vergelijkende studie van navigatietools voor mensen met een mentale beperking. Ik heb dit onderzoek gekozen vanwege een jongere broer die met het probleem kamt rond het navigeren vanzichzelf.

Ten slotte dank ik van harte als allereerste naar mijn co-promotor die vele versies naleeste en waardevolle feedback gaf. Dankzij zijn kennis en duiding kon de bacherlorproef in goede banen geleidt worden. Ook wil ik mijn mama bedanken voor het nalezen van de vele versies en het opleveren van waardevolle feedback.
