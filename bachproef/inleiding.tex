%%=============================================================================
%% Inleiding
%%=============================================================================

\chapter{\IfLanguageName{dutch}{Inleiding}{Introduction}}%
\label{ch:inleiding}

\section{\IfLanguageName{dutch}{Probleemstelling}{Problem Statement}}%
\label{sec:probleemstelling}

Mensen met een mentale beperking ondervinden bij het gebruik van bestaande mobiliteitstools
problemen, omdat deze steunen op een basisniveau lees- en schrijfvaardigheid en voldoende tijds- en ruimtebesef. Anders gezegd: met welke tool gebaseerd op visuele en/of auditieve signalen geraakt iemand die niet kan (klok)lezen zonder enige begeleiding van punt A naar punt B binnen het juiste tijdsbestek?


\section{\IfLanguageName{dutch}{Onderzoeksvraag}{Research question}}%
\label{sec:onderzoeksvraag}

Het globale objectief van dit onderzoek is te bestuderen welke tools het meest geschikt zijn voor deze doelgroep en op welke manier Augmented Reality (AR) ondersteuning deze efficiënter kan maken. Aan de hand van een requirementsanalyse en een vergelijkende studie van bestaande tools zijn duidelijke inzichten verworven omtrent de doelgroep en de criteria voor de meest geschikte navigatiemethode. Een aangepaste mobiliteitstool heeft immers een grote impact en meerwaarde op het vergroten van hun zelfstandigheid. 


\section{\IfLanguageName{dutch}{Onderzoeksdoelstelling}{Research objective}}%
\label{sec:onderzoeksdoelstelling}

De methodologie van dit onderzoek is opgedeeld in drie belangrijke fasen, elk essentieel voor het bereiken van de onderzoeksdoelstellingen: een literatuurstudie, een requirementsanalyse met long list en short list en de uitwerking van een Proof of Concept (PoC) aan de hand van de short list.

\section{\IfLanguageName{dutch}{Opzet van deze bachelorproef}{Structure of this bachelor thesis}}%
\label{sec:opzet-bachelorproef} 

Deze bachelorproef is vertrokken van een literatuurstudie naar de diverse navigatietechnologieën en heeft zo een uitgebreide blik geboden op de huidige stand van de techniek binnen het domein van navigatiesystemen en hun evolutie over de tijd. Bovendien is de betekenis van het begrip mentale beperking in detail toegelicht aan de hand van specifieke referenties. Het aftoetsen van die technologische vooruitgang aan tools die de levenskwaliteit van mensen met een mentale beperking kan verbeteren, is de essentiële schakel geweest tussen de algemene aspecten en de gerichte implementatie voor deze doelgroep.

Op basis hiervan zijn requirements geformuleerd voor een potentiële applicatie, die een vergelijking mogelijk maakten tussen diverse commercieel beschikbare navigatietools. In de Proof of Concept zijn twee gekozen navigatiemethoden beoordeeld op gebruiksgemak voor mensen met een mentale beperking. Hieruit zijn enkele aanbevelingen geformuleerd. 

Uiteindelijk is in de conclusie een synthese gemaakt van de resultaten en is de centrale onderzoeksvraag beantwoord, namelijk welke navigatietool is het best geschikt om aan te passen voor de doelgroep. Toekomstig onderzoek kan hierop verder bouwen en bepaalde requirements via specifieke aanpassingen implementeren.