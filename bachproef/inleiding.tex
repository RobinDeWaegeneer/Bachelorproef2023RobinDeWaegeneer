\chapter{\IfLanguageName{dutch}{Inleiding}{Introduction}}%
\label{ch:inleiding}
Navigatie voor mensen met een mentale beperking is uitdagend door hun beperkte lees- en schrijfvaardigheden en gebrekkig tijds- en ruimtebesef. Bestaande mobiliteitstools zijn hier vaak niet op aangepast, wat zelfstandig navigeren moeilijk maakt. Dit onderzoek richt zich op het identificeren van de meest geschikte navigatietool voor deze doelgroep en onderzoekt hoe Augmented Reality (AR) kan bijdragen aan gebruiksvriendelijkere navigatie.

\section{\IfLanguageName{dutch}{Probleemstelling}{Problem Statement}}%
\label{sec:probleemstelling}

Mensen met een mentale beperking ondervinden specifieke uitdagingen bij het gebruik van bestaande mobiliteitstools. De probleemstelling richt zich op het identificeren en aanpakken van de specifieke problemen die deze doelgroep ondervindt bij het navigeren met behulp van bestaande tools. Het doel is om te onderzoeken hoe een aangepaste navigatietool, ondersteund door AR, kan bijdragen aan een efficiëntere en gebruiksvriendelijkere navigatie-ervaring.

\section{\IfLanguageName{dutch}{Onderzoeksvraag}{Research question}}%
\label{sec:onderzoeksvraag}

De onderzoeksvraag is specifiek gericht op het identificeren van de beste navigatietool voor mensen met een mentale beperking. Hoe kan een tool deze mensen helpen om zonder begeleiding van punt A naar punt B te navigeren binnen het juiste tijdsbestek?

\section{\IfLanguageName{dutch}{Onderzoeksdoelstelling}{Research objective}}%
\label{sec:onderzoeksdoelstelling}

De doelstelling van dit onderzoek is om de meest geschikte navigatietool te identificeren voor mensen met een mentale beperking en te onderzoeken hoe AR ondersteuning deze tools efficiënter kan maken. Het einddoel is om een tool te ontwikkelen die de zelfstandigheid van deze doelgroep vergroot en advies te geven over welke bestaande tool het beste aansluit bij de noden van de doelgroep.

\section{\IfLanguageName{dutch}{Opzet van deze bachelorproef}{Structure of this bachelor thesis}}%
\label{sec:opzet-bachelorproef}

De rest van deze bachelorproef is als volgt opgebouwd:

In Hoofdstuk~\ref{ch:stand-van-zaken} wordt een overzicht gegeven van de stand van zaken binnen het onderzoeksdomein, op basis van een literatuurstudie.

In Hoofdstuk~\ref{ch:methodologie} wordt de methodologie toegelicht en worden de gebruikte onderzoekstechnieken besproken om een antwoord te kunnen formuleren op de onderzoeksvragen.

In Hoofdstuk~\ref{ch:proof-of-concept} wordt een Proof of Concept (PoC) ontwikkeld en getest om de praktische toepasbaarheid van de voorgestelde oplossingen te evalueren.

In Hoofdstuk~\ref{ch:conclusie}, tenslotte, wordt de conclusie gegeven en een antwoord geformuleerd op de onderzoeksvragen. Daarbij wordt ook een aanzet gegeven voor toekomstig onderzoek binnen dit domein.
