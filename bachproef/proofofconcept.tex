%%=============================================================================
%% Proof of Concept
%%=============================================================================

\chapter{\IfLanguageName{dutch}{Proof of Concept}{Proof of Concept}}%
\label{ch:proof-of-concept}

Dit deel van het onderzoek presenteert een Proof of Concept van een ondersteunende mobiliteitstool met Augemented Reality (AR) voor personen met een mentale beperking. Het doel van de Proof of Concept is aantonen in hoe verre een dergelijke mobiliteitstool mogelijk is. De ontwikkeling van deze applicatie en hieruit gehaalde resultaten worden besproken.

\begin{itemize}
    \item Locaties waar ze naartoe gaan onthouden en mooi tonen (eenvoudig/overzichtelijk) (foto van de locatie)
    \item De applicatie moet de routebegeleiding geven met en zonder AR
    \item De applicatie auditieve instructies en locatie voorlezen
    \item bBijsturen als het misgaat herbereken van een route
    \item Afgelegde pad bijhouden duur lengte 
    \item Sclera pictogrammen programmeren
    \item De app van de lijn of nmbs kunnen openen
    \item Navigatie methode kiezen trein bus fiets wandelen
    \item Het aantal stappen in het navigatieproces moet tot een minimum beperkt blijven.
    \item Aanpasbaarheid: Het systeem moet aanpasbaar zijn aan de individuele behoeften van de gebruiker, zoals voorkeursroutes en het vermijden van bepaalde obstakels.
\end{itemize}

%%TODO enquëte: Welke problemen ervaren personen met een mentale beperking bij het gebruik van bestaande mobiliteitstools?



