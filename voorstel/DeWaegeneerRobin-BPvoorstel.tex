%==============================================================================
% Sjabloon onderzoeksvoorstel bachproef
%==============================================================================
% Gebaseerd op document class `hogent-article'
% zie <https://github.com/HoGentTIN/latex-hogent-article>

% Voor een voorstel in het Engels: voeg de documentclass-optie [english] toe.
% Let op: kan enkel na toestemming van de bachelorproefcoördinator!
\documentclass{hogent-article}

% Invoegen bibliografiebestand
\addbibresource{voorstel.bib}

% Informatie over de opleiding, het vak en soort opdracht
\studyprogramme{Professionele bachelor toegepaste informatica}
\course{Bachelorproef}
\assignmenttype{Onderzoeksvoorstel}
% Voor een voorstel in het Engels, haal de volgende 3 regels uit commentaar
% \studyprogramme{Bachelor of applied information technology}
% \course{Bachelor thesis}
% \assignmenttype{Research proposal}

\academicyear{2023-2024} % TODO: pas het academiejaar aan

% TODO: Werktitel
\title{Vergelijkende studie van Augmented Reality ondersteunde mobiliteitstool voor personen met een mentale beperking: een proof of concept.}

% TODO: Studentnaam en emailadres invullen
\author{Robin De Waegeneer}
\email{robin.de.waegeneer@student.hogent.be}

% TODO: Medestudent
% Gaat het om een bachelorproef in samenwerking met een student in een andere
% opleiding? Geef dan de naam en emailadres hier
% \author{Yasmine Alaoui (naam opleiding)}
% \email{yasmine.alaoui@student.hogent.be}

% TODO: Geef de co-promotor op
\supervisor[Co-promotor]{T. Aelbrecht (HoGent, \href{mailto:thomas.aelbrecht@hogent.be}{thomas.aelbrecht@hogent.be})}

% Binnen welke specialisatierichting uit 3TI situeert dit onderzoek zich?
% Kies uit deze lijst:
%
% - Mobile \& Enterprise development
% - AI \& Data Engineering
% - Functional \& Business Analysis
% - System \& Network Administrator
% - Mainframe Expert
% - Als het onderzoek niet past binnen een van deze domeinen specifieer je deze
%   zelf
%
\specialisation{AI \& Data Engineering}
\keywords{Mobiliteitstool, mentale beperking, AR}

\begin{document}

\begin{abstract}
    Dit onderzoek richt zich op de integratie van Augmenten Reality (AR) voor de optimalisatie van interactieve mobiliteitstools voor mensen met een mentale beperking. 
    Mobiliteit is een belangrijk aspect van het dagelijkse leven en vormt een specifieke uitdaging voor mensen met een verstandelijke beperking. 
    Startpunt van het onderzoek zijn reeds bestaande tools die aan de hand van AR tot een volledig Proof-of-Concept (POC) zullen uitgewerkt worden. 
    Dankzij een bevraging van de stakeholders wil het onderzoek bijkomende inzichten creëren zowel omtrent de doelgroep als de concrete oplossingen. 
    Binnen de uitwerking van de POC is gepland om een groep kinderen met de ontworpen app te laten experimenteren om na te gaan hoe ze zich beter kunnen verplaatsen met de AR ondersteunde applicaties. 
    Uit de vergelijkende studie wordt verwacht dat specifieke AR functionaliteiten zullen worden geïdentificeerd voor de doelgroep. 
    Bovendien zal dankzij een testfase van de POC het effectieve gebruiksgemak worden beoordeeld. 
    Als conclusie zal deze studie een duidelijk lastenboek afleveren voor verdere ontwikkeling van een aangepaste mobiliteitstool voor mensen met een mentale beperking.
\end{abstract}

\tableofcontents

% De hoofdtekst van het voorstel zit in een apart bestand, zodat het makkelijk
% kan opgenomen worden in de bijlagen van de bachelorproef zelf.
\hyphenation{mo-bi-li-teits-tool}

%---------- Inleiding ---------------------------------------------------------

\section{Introductie}%
\label{sec:introductie}

    Mensen met een mentale beperking ondervinden bij het gebruik van bestaande mobiliteitstools problemen, omdat deze uitgaan van een basisniveau lees- en schrijfvaardigheid waarover zij niet beschikken. 
    Bovendien is in sommige gevallen ook hun tijds- en ruimtebesef gelimiteerd. 
    Het globale objectief van het onderzoek is een tool te ontwikkelen om deze doelgroep via Augmented Reality (AR) ondersteuning deze op een efficiënte en veilige manier te gebruiken. 
    De proof-of-concept (POC) wordt zo eenvoudig mogelijk ontwikkeld, zodat ze gestructureerd hun weg kunnen vinden met behulp van het openbaar vervoer. 
    Door bestaande tools te screenen, al dan niet gericht op onze doelgroep, en deze af te toetsen door bevraging van verschillende stakeholders, wordt een meer optimale tool in kaart gebracht. 
    Kort gezegd: hoe geraakt iemand die niet kan (klok)lezen zonder enige begeleiding van punt A naar punt B met visuele en/of auditieve signalen aan de hand van AR binnen het juiste tijdsbestek?
    We zijn ervan overtuigd dat deze mobiliteitstool een grote impact en meerwaarde zal hebben voor deze doelgroep door hun limieten te verleggen en zelfstandigheid te vergroten.

%---------- Stand van zaken ---------------------------------------------------

\section{State-of-the-art}%
\label{sec:state-of-the-art}

  De literatuurstudie zal een overzicht maken van de bestaande tools zoals de applicaties van NMBS\footnote{\url{https://www.belgiantrain.be/nl}}, De Lijn\footnote{\url{https://www.delijn.be/nl/}} en standaard routeplanners. 
  Deze applicaties en websites zijn gebaseerd op het principe dat de gebruiker zelfstandig een adres en tijdstip zelf kan ingeven en maken niet altijd gebruik van locatiebepaling, wat voor de doelgroep een duidelijke hinderpaal is.
    Daarnaast zullen diverse Augmented Reality (AR) methodes binnen de huidige state of the art vergeleken worden voor het verbeteren van de gebruikerservaring. 
    Mapbox AR\footnote{\url{https://www.mapbox.com/augmented-reality}} maakt gebruik van points of interest terwijl Google Maps AR \footnote{\url{https://arvr.google.com}} inzet op multidimensionele visualisatie om de gebruiker comfortabel aan te sturen tijdens het navigeren. 
    Met behulp van de camera van het mobiele apparaat worden real-time beelden van de omgeving vastgelegd, waarbij digitale routeaanwijzingen over de werkelijke beelden worden geprojecteerd. 
    Dit zorgt voor een intuïtieve en praktische navigatie-ervaring, vooral in stedelijke gebieden waar traditionele kaarten mogelijk minder effectief zijn. 
    Mapbox AR onderscheidt zich door zijn typische karakteristiek van geavanceerde aanpasbaarheid en integratiemogelijkheden. 
    Deze tool biedt ontwikkelaars een krachtig platform waarmee ze op maat gemaakte augmented reality-toepassingen kunnen creëren, variërend van navigatie tot locatiegebaseerde informatie. 
    Hierdoor hebben ontwikkelaars de flexibiliteit om kaartgegevens aan te passen, aangepaste overlays toe te voegen en de gebruikerservaring te optimaliseren voor specifieke doeleinden. 
    Beide tools vertegenwoordigen technologieën binnen het domein van augmented reality en kaartnavigatie. Ze tonen de evolutie aan van traditionele kaartapplicaties naar meer dynamische, op AR gebaseerde oplossingen.  
    Ook Koombea\footnote{\url{https://www.koombea.com}} verhoogt de gebruikerservaring en met AR City\footnote{\url{https://arcitygame.nl}} komt de locatie effectief tot leven dankzij bijkomende informatie. 
    Sygic\footnote{\url{https://www.sygic.com}} werd specifiek ontwikkeld voor auto GPS-systemen om de bestuurderservaring te optimaliseren, maar kan een toegevoegde waarde hebben wanneer de gebruiker tijdens een busrit het traject mee wil opvolgen. 
    Functionaliteiten van elk deze tools zullen gescreend worden om te bepalen of zij in aanmerking komen voor integratie in de proof-of-concept (POC). 
    De implementatie van locatiegevoelige informatie in combinatie met de AR visualisatie neemt diverse barrières weg voor de doelgroep. 
    Zij hoeven immers niet het adres te kennen van hun startpunt. Het eindpunt kunnen ze bv. omschrijven via een pictogram zoals deze gebruikt worden in Sclera\footnote{\url{https://www.sclera.be/nl/picto/overview}}. 
    Deze pictogrammen worden standaard aangeleerd bij onze doelgroep ter bevordering van hun communicatie en visualisatie van hun dagplanning.
    Naast het openbaar vervoer kunnen andere systemen zoals Dott\footnote{\url{https://ridedott.com/nl/}} en Villo\footnote{\url{https://www.villo.be/nl/home}} worden overwogen omdat dergelijke fietsen en steps verspreid staan in diverse steden. 
    Ze bieden een bijkomende optie voor het verplaatsen van punt A naar punt B op een efficiënte manier.
    Op hardwaregebied ontwikkelde Google in 2014 een smart device genaamd Google Glasses. 
    Deze brillen hadden als doel om AR een extra boost te geven aan de hand van een concreet concept. 
    Het product kreeg een tweede versie in 2017, als ondernemingseditie, maar had geen succes en eerder dit jaar in maart 2023 kondigde Google aan dat ze het project stopzetten \autocite{Gvora2023}.
    Tot slot geldt dat de doelgroep die niet in staat is te lezen of schrijven, wel gebruik kan maken van text-to-speech functionaliteit om de gewenste locatie te omschrijven. 
    De tool moet met andere woorden ook deze stemherkenning als ondersteuning bieden. 
    Dankzij artificiële intelligentie (AI) wordt dan de beste route berekent en tevens de beste manier om daar gemakkelijk te geraken \autocite{Soni2023a}. 
    Het gebruik van datastructuren zal toelaten deze locatie informatie te linken aan het routepatroon \autocite{Ruta2010}. 
    Bovendien kunnen deze patronen potentiële knelpunten identificeren, hulp bieden bij het optimaliseren van routes als er verkeerswerken of files zijn. 
    De noodzaak van dit systeem zijn realtime gegevens waarbij samenwerking met belanghebbende belangrijk is \autocite{Ciravegna2018}. 
    Het opzetten van een meta model kan begeleiders van mensen met een matige mentale beperking helpen met het navigeren naar een juiste plaats. 
    Door gegevensverzameling kan er aan de hand van algoritmes voorspellingen gemaakt worden wat voor deze persoon de favoriete manier van verplaatsen is of favoriete route is om te volgen naar het werk \autocite{Stepanov2003}. 
    Als laatste kan er gebruik gemaakt worden van Internet of Things (IoT) tools voor het slim communiceren tussen verschillende apparatuur wat kan leiden tot nog betere prestaties in verplaatsing en routeberekening \autocite{Fatnassi2015}.

% Voor literatuurverwijzingen zijn er twee belangrijke commando's:
% \autocite{KEY} => (Auteur, jaartal) Gebruik dit als de naam van de auteur
%   geen onderdeel is van de zin.
% \textcite{KEY} => Auteur (jaartal)  Gebruik dit als de auteursnaam wel een
%   functie heeft in de zin (bv. ``Uit onderzoek door Doll & Hill (1954) bleek
%   ...'')

%---------- Methodologie ------------------------------------------------------
\section{Methodologie}%
\label{sec:methodologie}

Herschrijven van de methodologie

%---------- Verwachte resultaten ----------------------------------------------
\section{Verwacht resultaat, conclusie}%
\label{sec:verwachte_resultaten}

Uit de screening van bestaande tools zullen nuttige functies worden gebruikt als bouwstenen voor de proof-of-concept (POC) van dit onderzoek. 
Dankzij de inzichten verworven in de bevraging van de noden en gesprekken met diverse stakeholders zal deze POC zo functioneel mogelijk zijn voor de specifieke doelgroep. 
Tot slot zal dankzij een experimentele fase nagegaan worden hoe efficiënt de concrete uitwerking is en welke positieve impact AR implementatie kan hebben.
In essentie kan worden gesteld dat er al enkele bruikbare tools ontwikkeld zijn, maar deze niet (volledig) aan de noden voldoen voor mensen met een matige mentale beperking. Deze hulpmiddelen vereisen nog steeds een redelijke kennis in het gebruik. Daarom wordt, na bevragingen, een POC opgesteld aan de hand van verworven inzichten. De bekomen resultaten van de literatuurstudie, in combinatie met de bevragingen en gebouwde POC, zullen een duidelijk beeld schetsen voor de samenwerking met een hardwarebedrijf of AR bedrijf. 
Als resultaat hierop wordt meteen ook nagedacht over extra functies en integraties die dan samengevat worden als opsomming met latere innovaties.



\printbibliography[heading=bibintoc]

\end{document}